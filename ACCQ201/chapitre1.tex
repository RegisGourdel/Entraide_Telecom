\subsection{Relations d'équivalence et structures quotient}

	\begin{defn}
		Une \textbf{relation} $\rel$ sur un ensemble $E$ est la donnée d'une partie $\rel \subset E \times E$.
		On écrit $x \rel y$ si $(x,y) \in \rel$.
	\end{defn}

	\begin{defn}
		Une relation $\rel$ sur $E$ est dite :
		\begin{itemize}
			\item[\textbullet] réflexive si $\forall x \in E, x \rel x$,
			\item[\textbullet] symétrique si $\forall x, y \in E, \left( x \rel y \right) \implies \left( y \rel x \right)$,
			\item[\textbullet] transitive si $\forall x, y, z \in E, \left( x \rel y \text{ et } y \rel z \right) \implies x \rel z$
		\end{itemize}
		et on dit que c'est une relation d'équivalence si ces trois conditions sont vérifiées.
		Dans ce cas on note aussi $x \sim_\rel y$ ou encore $x \equiv y \mod \rel$.
	\end{defn}
	
	\begin{defn}
	Soit $\sim$ une relation d'équivalence sur $E$ et $A \subset E$.
	On dit que $A$ est une classe d'équivalence pour la relation $\sim$ si $A$ est non vide, $\forall x,y \in A, x \sim y$ et $\forall x \in A, \forall y \not\in A, x \not\sim y$.
	\end{defn}

	\begin{note}
		On note $E / \sim$ l'ensemble des classes d'équivalences pour $\sim$, appelé \textbf{ensemble quotient} de $E$ par $\sim$.
	\end{note}

	\begin{pop}
		Les classes d'équivalence forment une partition de $E$.
	\end{pop}

	\begin{cor}
		On a $\abs{E} = \sum_{A \in E/\sim} \abs{A}$.
	\end{cor}

	On dispose de la projection canonique de $E$ sur $E/\sim$, $\pi \colon x \mapsto \bar{x} = \{ y \in E \mid x \sim y \}$, où $x$ est un représentant de $\bar{x}$.
	On dit que $\mathcal{S} \subset E$ est un \textbf{système de représentants} si $\pi_{|\mathcal{S}} \colon \mathcal{S} \overset{\sim}{\to} E/\sim$.

	\begin{thm}
		Soit $f \colon E \to F$ une application.
		On a équivalence entre les deux assertions suivantes :
		\begin{enumerate}
			\item $f$ est compatible à $\sim$, i.e. $\forall x,y \in E, x \sim y \implies f(x) = f(y)$
			\item $\exists g \colon E/\sim \to F, f = g \circ \pi$.
		\end{enumerate}
		Si ces conditions sont vérifiées, cette application $g$ est unique.
		On dit qu'elle est l'application déduite de $f$ par passage au quotient par $\sim$.
	\end{thm}

	\begin{defn}
		Un \textbf{groupe} $(G,*,e)$ est la donnée de $G$ non vide, $*$ une loi de composition interne et $e \in G$ tels que $*$ est associative, $e$ est neutre et tout élement est inversible.
		On dit que ce groupe est \textbf{abélien} si $*$ est commutative.
	\end{defn}

	\begin{defn}
		Un sous-ensemble $H$ du groupe $G$ est appelé \textbf{sous groupe} si : $e \in H$, il est stable par inversion et par composition.
	\end{defn}

	\begin{pop}
		Une intersection quelconque de sous-groupes de $G$ est encore un sous-groupe de $G$.
	\end{pop}

	\begin{pop}
		Soit $G$ un groupe et $S \subset G$.
		Notons $\langle S \rangle \subset G$ l'intersection de tous les sous-groupes de $G$ qui contiennent $S$.
		Alors $\langle S \rangle$ est un sous-groupe de $G$ contenant $S$ et c'est le plus petit d'entre eux.
		On l'appelle \textbf{sous-groupe} engendré par $S$ dans $G$.
	\end{pop}

	\begin{pop}
		On a aussi $S = \{ s_1^{m_1} * \cdots * s_r^{m_r} \mid r \in \N, s_i \in S, m_i \in \Z \}$.
	\end{pop}

	\begin{pop}
		Soit $H$ un sous-groupe de $G$.
		On définit $x \sim y \iff x^{-1}y \in H$.
		Alors $\sim$ est une relation d'équivalence et $G/\sim$ est noté $G/H$ $(x \mod H) = xH$.
	\end{pop}

	\begin{defn}
		Soit $H$ un sous-groupe de $G$.
		On définit l'\textbf{indice} de $H$ dans $G$ par $[G : H] = \abs{G/H}$ (éventuellement infini).
	\end{defn}

	\begin{pop}
		Soit $G$ un groupe fini.
		Alors tout sous-groupe $H$ de $G$ est fini, d'indice fini et on a $\abs{G} = \abs{H} \cdot [G : H]$.
	\end{pop}


\subsection{Action d'un groupe sur un ensemble}

	\begin{defn}
		Une action de $(G,*,e)$ sur un ensemble $X$ est la donnée d'une application
		$\begin{array}{ccc} G \times X & \to & X \\ (g,x) & \mapsto & g \cdot x \end{array}$
		telle que $\forall x \in X, e \cdot x = x$ et $\forall g, g' \in G, \forall x \in X, g \cdot (g' \cdot x) = (g * g') \cdot x$.
	\end{defn}

	\begin{defn}
		Soit $G$ agissant sur $X$.
		Pour tout élément $x \in X$, on définit son stabilisateur $G_x = \{ g \in G \mid g \cdot x = x \}$ et son orbite $O_x = G \cdot x = \{ g \cdot x \mid g \in G \}$.
	\end{defn}

	\begin{pop}
		Le stabilisateur $G_x$ est un sous-groupe de $G$.
	\end{pop}

	\begin{pop}
		La relation $\sim$ définie par $(x \sim x') \iff (\exists g \in G, x' = g \cdot x)$ est une relation d'équivalence.
		Les orbites de l'action sont alors précisément le classes d'équivalences pour $\sim$.
	\end{pop}

	\begin{ex}
		Des actions classiques de $H$ sur $G$, avec $H$ sous-groupe de $G$, sont :
		\begin{enumerate}
			\item translation à gauche, $(h,x) \mapsto x h^{-1}$,
			\item translation à droite, $(h,x) \mapsto hx$,
			\item conjugaison, $(h,x) \mapsto h x h^{-1}$.
		\end{enumerate}
	\end{ex}

	\begin{pop}
		Soit $G$ agissant sur $X$.
		Alors $\forall x \in X$, $g \mapsto g \cdot x$ induit par passage au quotient une bijection $G/G_x \overset{\sim}{\to} O_x$ et en particulier $\abs{O_x} = [G : G_x]$.
	\end{pop}

	\begin{thm}[Formule de Burnside]
		Soit $G$ un groupe fini agissant sur $X$ fini.
		Alors
		$$\abs{X/G} = \frac{1}{\abs{G}} \sum_{g \in G} \abs{\{ x \in X \mid g \codt x = x \} }$$
		autrement dit le nombre d'orbites de l'action est égal à l'espérance du nombre de points fixes d'un élément aléatoire de $G$.
	\end{thm}


\subsection{Morphismes}

	\begin{defn}
		Un \textbf{(homo)morphisme} de $(G,*,e)$ dans $(G',*',e')$ est une application $f \colon G \to G'$ telle que : $f(e) = e'$, $\forall x \in G, f(x^{-1}) = f(x)^{-1}$, $\forall x,y \in G, f(x * y) = f(x) *' f(y)$.
		Pour $G = G'$ c'est un \textbf{endomorphisme}, si $f$ est bijective c'est un \textbf{isomorphisme} et pour les deux à la fois c'est un \textbf{automorphisme}.
	\end{defn}

	\begin{pop}
		Les images directes et réciproques de sous-groupes par un morphisme de groupes sont des sous-groupes.
	\end{pop}

	\begin{defn}
		Soit $f \colon G \to G'$ un morphisme de groupes.
		Alors
		\begin{enumerate}
			\item $\Ker(f) = f^{-1}(e')$ est un sous-groupe de $G$ appelé \textbf{noyau} de $f$,
			\item $\Im(f) = f(G)$ est un sous-groupe de $G'$ appelé \textbf{image} de $f$.
		\end{enumerate}
	\end{defn}

	\begin{pop}
		Un morphisme $f \colon G \to G'$ est injectif si et seulement si $\Ker(f) = \{ e \}$.
	\end{pop}


\subsection{Groupe quotient d'un groupe abélien par un sous-groupe}

	\begin{lem}
		Soit $H$ s-g de $(G,+,0)$ abélien.
		On munit $G/H$ d'une structure de groupe abélien avec la loi $+$ telle que $\bar{a} + \bar{b} = (a + H) + (b + H) = (a + b) + H = \overline{a + b}$.
		L'ensemble quotient muni de cette loi est appelé \textbf{groupe quotient} de $G$ par $H$.
	\end{lem}
