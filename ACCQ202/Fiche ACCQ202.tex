\documentclass[a4paper,11pt]{article}

\usepackage[utf8]{inputenc}
\usepackage[T1]{fontenc}
\usepackage{lmodern}
\usepackage{amsthm} %ou \usepackage{ntheorem}
\usepackage[top=1.5cm, bottom=2.2cm, left=2cm, right=2cm]{geometry}
\usepackage{fancyhdr}
\usepackage{graphicx}
\usepackage{draftwatermark}
\usepackage{multicol}
\usepackage{enumerate}
\usepackage{titlesec}
\usepackage{alltt}
\usepackage[svgnames]{xcolor}
\usepackage[francais]{babel}
\pagestyle{fancy}


%
% Création du filigrane
%
\SetWatermarkText{\includegraphics{../logo_blanc.png}}
\SetWatermarkScale{0.5}
\SetWatermarkAngle{0}


%
% Suivent les macros qui donnent la mise en forme des théorèmes, définitions, etc...
%
\newtheoremstyle{persoth}% name
{2pt}%Space above
{2pt}%Space below
{\itshape}%Body font
{}%Indent amount
{\bf}%Theorem head font
{.}%Punctuation after theorem head
{.5em}%Space after theorem head 2
{}%

\newtheoremstyle{persodef}% name
{2pt}%Space above
{2pt}%Space below
{}%Body font
{}%Indent amount
{\bf}%Theorem head font
{.}%Punctuation after theorem head
{.5em}%Space after theorem head 2
{}%

\theoremstyle{persoth}% default
\newtheorem*{thm}{\noindent\textcolor{Crimson}{Th}}
\newtheorem*{lem}{\noindent\textcolor{MediumVioletRed}{Lem}}
\newtheorem*{pop}{\noindent\textcolor{FireBrick}{Prop}}
\newtheorem*{cor}{\noindent\textcolor{Brown}{Cor}}

\theoremstyle{persodef}
\newtheorem*{defn}{\noindent\textcolor{magenta}{Def}}
\newtheorem*{conj}{Conjecture}

\theoremstyle{remark}
\newtheorem*{rem}{\noindent\textcolor{Teal}{Rem}}
\newtheorem*{ex}{\noindent\textcolor{DarkOrange}{Ex}}
\newtheorem*{note}{\noindent\textcolor{RoyalBlue}{Not}}
\newtheorem*{danger}{\textcolor{green}{Attention}}
\newtheorem*{voc}{\textcolor{DarkGreen}{Voc}}
\newtheorem*{hyp}{\textcolor{OrangeRed}{Hyp}}

%\newcommand\demo{\begin{proof}[\textit{Démonstration}]}


%
% Redéfinition de caractéristiques sur le format des titres de sections etc...
%
\titlespacing{\section}{0pt}
	{8pt}{3pt}

\titleformat{\subsection}
	[hang]% style : hang, display, runin, leftmargin, ...
	{\normalsize\bfseries\sffamily}% fonte numéro + titre
	{\thesubsection}% numéro
	{1em}% espace entre le numéro et le titre
	{}% fonte titre

\titlespacing{\subsection}{0pt}
	{6pt}{1pt}


\renewcommand*\rmdefault{ppl}
\date{}

\usepackage{pgf, tikz}
\usetikzlibrary{arrows,chains,matrix,positioning,scopes,calc,intersections,through,backgrounds}
\makeatletter
\tikzset{join/.code=\tikzset{after node path={%
\ifx\tikzchainprevious\pgfutil@empty\else(\tikzchainprevious)%
edge[every join]#1(\tikzchaincurrent)\fi}}}
\makeatother
%
\tikzset{>=stealth',every on chain/.append style={join},
         every join/.style={->}}
\tikzstyle{labeled}=[execute at begin node=$\scriptstyle,
   execute at end node=$]

\fancyhf{}
\renewcommand{\headrulewidth}{0pt}
\fancyfoot[C]{\tiny Che Bedara - BDE Télécom ParisTech - Régis}
\fancyfoot[RO]{\thepage}
\fancyfoot[LE]{\thepage}

\usepackage{amsthm} %ou \usepackage{ntheorem}
\usepackage{amsmath}
\usepackage{amssymb}
\usepackage{mathrsfs}
\usepackage{amsfonts}

\definecolor{vert}{rgb}{0,0.6,0}


% Une macro pour obtenir de grandes fractions dans les formules en ligne.
\def\frc#1#2{\displaystyle{#1\over#2}}

% Une macro pour les vecteurs qui donne de meilleurs résultats que \overrightarrow.
\def\vect#1{%
	\vbox{\lineskip=-.04em\baselineskip=0pt
	\halign{##\cr
	\leaders\hbox{$\scriptstyle{-}$\kern-.4em}\hfil$\scriptstyle{\rightarrow}$\cr
	$#1$\cr}}}

% Majuscules d'anglaise.
\DeclareSymbolFont{rsfscript}{U}{rsfs}{m}{n}
\DeclareSymbolFontAlphabet{\mathrsfs}{rsfscript}
\newcommand\scr{\mathrsfs}


% Des macros pour les notations usuelles.
\newcommand{\ensemblenombre}[1]{\mathbf{#1}}
\newcommand{\N}{\ensemblenombre{N}}
\newcommand{\Z}{\ensemblenombre{Z}}
\newcommand{\Q}{\ensemblenombre{Q}}
\newcommand{\R}{\ensemblenombre{R}}
\newcommand{\C}{\ensemblenombre{C}}
\newcommand{\K}{\ensemblenombre{K}}
\newcommand{\U}{\ensemblenombre{U}} % Groupe du cercle unité complexe
\newcommand{\T}{\ensemblenombre{T}} % Tore [0;2π[
\newcommand{\mes}{\ensemblenombre{M}} % Espace de mesures
\newcommand\M{\mathfrak{M}}
\newcommand\E{\mathcal{E}}
\newcommand\parties{\mathcal{P}}
\newcommand\GL{\mathcal{GL}}
\newcommand\Sym{\mathcal{S}}
\newcommand\aSym{\mathcal{A}}
\newcommand\proba{\mathbf{P}}
\newcommand\esp{\mathbf{E}}
\newcommand\Orth{\mathcal{O}}
\newcommand\cont{\mathcal{C}}
\newcommand\normale{\mathcal{N}}
\newcommand\li{[\![}
\newcommand\ri{]\!]}
\newcommand{\diff}{\mathop{}\mathopen{}\mathrm{d}}
\newcommand{\abs}[1]{\left\lvert#1\right\rvert}
\newcommand{\norme}[1]{\left\lVert#1\right\rVert}
\newcommand{\norm}[1]{\left\lVert#1\right\rVert}
\newcommand{\transp}[1]{\ensuremath{#1^\mathsf{T}}}
\newcommand{\scal}[2]{\left\langle #1 \mid #2 \right\rangle}
\newcommand{\compl}[1]{{#1}^{\mathcal{C}}} % symbole du complémentaire en exposant
\newcommand\indep{\protect\mathpalette{\protect\independenT}{\perp}} % symbole d'indépendance en probas
\def\independenT#1#2{\mathrel{\rlap{$#1#2$}\mkern3mu{#1#2}}}
\newcommand\rel{\mathcal{R}} % Pour les relations binaires
\newcommand\rar{\rightarrow}
\newcommand\lar{\leftarrow}

% Notation d'ensembles en algèbre
\newcommand\Hom{\mathrm{Hom}}
\newcommand\End{\mathrm{End}} % Endomorphismes
\newcommand\Isom{\mathrm{Isom}} % Isométries
\newcommand\Aut{\mathrm{Aut}} % Automorphismes
\newcommand\Int{\mathrm{Int}}


%
% Intervalles
\newcommand{\intervalle}[4]{\mathopen{#1}#2\mathclose{}\mathpunct{};#3\mathclose{#4}}
\newcommand{\inff}[2]{\intervalle{[}{#1}{#2}{]}}
\newcommand{\inof}[2]{\intervalle{]}{#1}{#2}{]}}
\newcommand{\info}[2]{\intervalle{[}{#1}{#2}{[}}
\newcommand{\inoo}[2]{\intervalle{]}{#1}{#2}{[}}
\newcommand{\iniff}[2]{\intervalle{[\![}{#1}{#2}{]\!]}}

% Secondes définitions d'intervalles, taille ajustable
\newcommand{\genericinterval}[4]
{
	\mathopen{}\mathclose{\left#1#2\mathclose{}\mathpunct{};#3\right#4}
}
\newcommand{\intff}[2]{\genericinterval{[}{#1}{#2}{]}}
\newcommand{\intoo}[2]{\genericinterval{]}{#1}{#2}{[}}
\newcommand{\intof}[2]{\genericinterval{]}{#1}{#2}{]}}
\newcommand{\intfo}[2]{\genericinterval{[}{#1}{#2}{[}}
\newcommand{\intiff}[2]{\genericinterval{[\![}{#1}{#2}{]\!]}}


%
% Opérateurs mathématiques
%
\DeclareMathOperator{\card}{Card}
\DeclareMathOperator{\tr}{Tr}
\DeclareMathOperator{\Ker}{Ker}
\DeclareMathOperator{\Vect}{Vect}
\DeclareMathOperator{\Span}{Span} % Version anglaise de Vect
\DeclareMathOperator{\indic}{\mathbf{1}}
\DeclareMathOperator{\argth}{argth}
\DeclareMathOperator{\Id}{Id}
\DeclareMathOperator{\Gram}{Gram}
\DeclareMathOperator{\diag}{diag}
\DeclareMathOperator{\Mat}{Mat}
\DeclareMathOperator{\Sp}{Sp}
\DeclareMathOperator{\im}{Im}
\DeclareMathOperator{\Cov}{Cov}
\DeclareMathOperator{\Var}{Var}
\DeclareMathOperator{\Supp}{Supp}
\DeclareMathOperator{\sinC}{sinC}
\DeclareMathOperator{\sgn}{sgn}
\DeclareMathOperator{\argmin}{arg\, min}
\DeclareMathOperator{\cond}{cond} % Fonction conditionnement en optimisation linéaire
\DeclareMathOperator{\proj}{proj}
\DeclareMathOperator{\Bias}{Bias}
\DeclareMathOperator{\MSE}{MSE}
\DeclareMathOperator{\pgcd}{pgcd}
\DeclareMathOperator{\ppcm}{ppcm}
\DeclareMathOperator{\KL}{KL} % Kullback-Leibler divergence


% Pour des symboles « inférieur ou égal », « supérieur ou égal », « ensemble vide » et « parallèles » conformes aux usages français.
\DeclareSymbolFont{AmsA}{U}{msa}{m}{n}
\SetSymbolFont{AmsA}{bold}{U}{msa}{b}{n}
\DeclareMathSymbol\leq\mathrel{AmsA}{"36}
\DeclareMathSymbol\geq\mathrel{AmsA}{"3E}
\DeclareSymbolFont{AmsB}{U}{msb}{m}{n}
\SetSymbolFont{AmsB}{bold}{U}{msb}{b}{n}
\DeclareMathSymbol\emptyset\mathord{AmsB}{"3F}
\def\parallel{\mathrel{/\!/}}


\title{\vspace{-1.2cm} \textbf{ACCQ 202 - Information theory}}


\begin{document}

\maketitle

\vspace{-1.5cm}

\section{Source coding}

	\begin{defn}[Le modèle de transmission]
	$$\overset{D_1,\ldots,D_k}{\longrightarrow}
		\text{Émetteur } f
	\overset{X_1,\ldots,X_n}{\longrightarrow}
		\text{Canal}
	\overset{Y_1,\ldots,Y_n}{\longrightarrow}
		\text{Récepteur } g
	\overset{\hat{D}_1,\ldots,\hat{D}_k}{\longrightarrow}$$
\end{defn}

\begin{voc}
	\begin{itemize}
		\item[\textbullet] Bits d'information : $\mathbf{D} = (D_1,\ldots,D_k)$, représentent les données à transmettre, supposés iid $\mathcal{B} \left( \frac{1}{2} \right) $.
		\item[\textbullet] Transmission en bloc : les $k$ bits d'information sont envoyés sur un bloc de $n$ utilisations du canal.
		\item[\textbullet] Émetteur : associe $\mathbf{X} = (X_1,\ldots,X_kn)$ à $\mathbf{D} = (D_1,\ldots,D_k)$, supposé déterministe et avec $f$ injective.
		\item[\textbullet] Récepteur : associe $\mathbf{\hat{D}} = \left( \hat{D}_1,\ldots,\hat{D}_k \right)$ à $\mathbf{Y} = (Y_1,\ldots,Y_kn)$, supposé déterministe.
		\item[\textbullet] Erreur : cas où $\left( \hat{D}_1,\ldots,\hat{D}_k \right) \neq (D_1,\ldots,D_k)$.
	\end{itemize}
\end{voc}


\subsection{Les canaux}

	\begin{defn}[Canaux discrets sans mémoire (\textbf{DMC})]
		Un DMC est complètement caractérisé par le triplet $\left( \mathcal{X}, \mathcal{Y}, \proba_{X \mid Y}(\cdot \mid \cdot) \right)$ où
		\begin{itemize}
			\item $\mathcal{X}$ est un alphabet fini contenant toutes les valeurs possibles à l'entrée du DMC,
			\item $\mathcal{Y}$ est un alphabet fini contenant toutes les valeurs possibles à la sortie du DMC,
			\item $\proba_{X \mid Y}(\cdot \mid \cdot)$ est une loi de probabilité conditionnelle, dite loi de transition, décrivant comment une sortie $Y_t$ est obtenue à partir d'une entrée $x_t$.
		\end{itemize}
	\end{defn}

	\begin{defn}[Canal binaire symétrique (\textbf{BSC})]
		On a $\mathcal{X} = \mathcal{Y} = \{ 0,1 \}$ et $\forall t \in \intiff{1}{n}$, $\proba(Y_t = x_t) = p \in \inff{0}{1}$ et $\proba(Y_t \neq x_t) = 1 - p$.
		On peut toujours supposer $p < \frac{1}{2}$.
	\end{defn}

	\begin{center}
	\begin{tikzpicture}
	\matrix(m)[matrix of math nodes,row sep=3em,column sep=4em,minimum width=2em]
	{
		0 & 0 \\
		1 & 1\\};
	\path[-stealth]
		(m-1-1) edge node [above] {$1 - p$} (m-1-2)
				edge node [above] {$p$} (m-2-2)
		(m-2-1) edge node [below] {$p$} (m-1-2)
				edge node [below] {$1 - p$} (m-2-2);
	\end{tikzpicture}
	\end{center}

	\begin{defn}[Canal binaire à effacement (\textbf{BEC})]
		On a $\mathcal{X} = \{ 0,1 \}$ et $\mathcal{Y} = \{ 0,1,\Delta \}$ où $\Delta$ représente un effacement.
		Pour tout $t \in \intiff{1}{n}$, $\proba(Y_t = x_t) = \epsilon \in \inff{0}{1}$ et $\proba(Y_t = \Delta) = 1 - \epsilon$.
	\end{defn}

	\begin{center}
	\begin{tikzpicture}
	\matrix(m)[matrix of math nodes,row sep=3em,column sep=4em,minimum width=2em]
	{
		0 & 0 \\
		  & \Delta\\
		1 & 1\\};
	\path[-stealth]
		(m-1-1) edge node [above] {$\epsilon$} (m-1-2)
				edge node [left] {$1 - \epsilon$} (m-2-2)
		(m-3-1) edge node [left] {$1 - \epsilon$} (m-2-2)
				edge node [below] {$\epsilon$} (m-3-2);
	\end{tikzpicture}
	\end{center}

\subsection{Codage par des codes en bloc}

	\begin{defn}
		Un \textbf{code en bloc} $\mathcal{C}$ de longueur $n$ sur un alphabet $\mathcal{X}$ est un sous-ensemble de $\mathcal{X}^n$, c'est l'ensemble d'arrivée de $f$.
		Les éléments de $\mathcal{C}$ sont appelés les mots de code de $\mathcal{C}$.
	\end{defn}

	Principe : $\underset{k\ \text{bits}}{[\quad m\quad ]} \longrightarrow \underset{n\ \text{bits}}{[\ c \in \mathcal{C}\ ]}$ avec $n > k$.

	\begin{defn}
		Le \textbf{rendement} (binaire) d'un code en bloc $\mathcal{C}$ de longueur $n$, aussi appelé taux de codage, est $R = \frac{\log_2(\abs{\mathcal{C}})}{n}$ où $\abs{\mathcal{C}}$ est le nombre de mots du code $\abs{\mathcal{C}}$.
	\end{defn}


\subsection{Distances}

	\begin{defn}
		Poids de Hamming pour $\mathbf{x} = (x_1,\ldots,x_n)$ : $w_H(\mathbf{x}) = \card \left( \left\{ x_i \neq 0 \right\} \right)$.
	\end{defn}

	\begin{defn}
		La \textbf{distance de Hamming} entre deux mots $\mathbf{x}$ et $\mathbf{\hat{x}}$ est donnée par $d_H(\mathbf{x},\mathbf{\hat{x}}) = w_H(\mathbf{x} - \mathbf{\hat{x}})$.
	\end{defn}

	\begin{pop}
		La distance de Hamming est bien une distance (symétrie, positivité et inégalité triangulaire).
	\end{pop}

	\begin{defn}
		La \textbf{distance minimale} du code en bloc $\mathcal{C}$ est
		$d_{\min}(\mathcal{C}) =
			\min_{ \mathbf{c} \neq \mathbf{\hat{c}} }
			d_H(\mathbf{c},\mathbf{\hat{c}})$.
	\end{defn}


\subsection{Décodage}

	On décompose la fonction de décodage en deux étapes : $g = g_2 \circ g_1$, $g_1$ trouve pour toute observation $Y$ le mot de code $\mathbf{\hat{c}} \in \mathcal{C}$ qui paraît le plus probable et $g_2 = f^{-1}$ produit la suite des bits détectés $\hat{D}_1,\ldots,\hat{D}_k$ qui est associée à $\hat{c}$.

	\begin{defn}
		Soit $g_1$ fixée. La région de décision associée à $c \in \mathcal{C}$ est $\Omega_{\mathbf{c}} := g_1^{-1}(c)$.
		Ces régions forment une partition de $\mathcal{Y}^n$.
	\end{defn}

	On a $P_e := \proba(\mathbf{\hat{C}} \neq \mathbf{C})$ la probabilité d'erreur et $P_c := \proba(\mathbf{\hat{C}} = \mathbf{C})$ la probabilité de succès.

	\begin{pop}[Optimalité de la règle de maximum vraisemblance, \textbf{maximum likelihood}]
		Si les mots de code sont équiprobables en entrée, minimiser $P_e$ revient à choisir $\hat{c} \in \argmax_{x \in \mathcal{C}} p(Y \mid x)$.
	\end{pop}

	\begin{pop}[\textbf{Règle du voisin le plus proche}]
		Dans le cas d'un BSC on a
		$\forall p \in \intff{0}{\frac{1}{2}}, \proba(\mathbf{Y} = \mathbf{y} \mid \mathbf{C} = \mathbf{c}) = (1 - p)^n \left( \frac{p}{1 - p} \right)^{d_H(\mathbf{y},\mathbf{c})}$
		donc minimiser $P_e$ revient à trouver le mot de code $\mathbf{c} \in \mathcal{C}$ qui minimise $d_H(\mathbf{y},\mathbf{c})$.
	\end{pop}

	\begin{voc}
		On dit qu'un code en bloc $\mathcal{C}$ corrige $t$ erreurs si il existe un décodeur qui permet de corriger toutes les configurations d'erreurs dont le nombre est inférieur ou égal à $t$.
	\end{voc}

	\begin{pop}[\textbf{Capacité de correction} d'un code]
		Le décodeur décide toujours du bon mot $\mathbf{\hat{c}} = \mathbf{c}$ lorsque $2 d_H(\mathbf{c},\mathbf{y}) < d_{\min}$.
		Donc le code peut corriger $t = \left\lfloor \frac{d_{\min} - 1}{2} \right\rfloor$ erreurs.
	\end{pop}

	Lorsque l'on fait de la détection d'erreur, on a $g_1 \colon \mathcal{Y}^n \to \mathcal{C} \cup \Delta$.
	La question est alors : est-ce que le mot reçu est bien égal au mot envoyé ?
	Dans le cas où $d_H(\mathbf{c},\mathbf{y}) = l \geq 1$ et le décodeur produit $\Delta$, on dit que le décodeur a détecté $l$ erreurs.

	\begin{voc}
		Dans le cas d'un BEC, on dit qu'un code en bloc $\mathcal{C}$ détecte $t$ erreurs si il existe un décodeur qui permet de corriger toutes les configurations d'erreurs dont le nombre est inférieur ou égal à $t$.
	\end{voc}

	\begin{pop}[\textbf{Capacité de détection} d'un code]
		Un code en bloc est capable de détecter $t' = d_{\min} - 1$ erreurs.
	\end{pop}
	Il suffit pour cette détection de déclarer $\Delta$ dès que $\mathbf{y} \not\in \mathcal{C}$.

\section{Entropie et questionnement}

	Let $(\Omega, \mathcal{F}, \proba)$ be a probability space.

\begin{thm}[$\pi$ - $\lambda$ theorem]
	If $\mathcal{A} \subset \mathcal{C}$ with $\mathcal{A}$ a $\pi$-system and $\mathcal{C}$ a $\lambda$-system, then $\sigma(\mathcal{A}) = \mathcal{C}$.
\end{thm}

\begin{thm}[Characterization of probability measures]
	Let $\mathcal{C}$ be a $\pi$-system on $\Omega$ and $\mathcal{F} = \sigma(\mathcal{C})$ the smallest $\sigma$-field containing $\mathcal{C}$.
	Then a probability measure $\mu$ on $(\Omega, \mathcal{F})$ is uniquely characterized by $\mu(A)$ on $A \in \mathcal{C}$.
\end{thm}

\begin{note}
	For $p > 0$, we denote by $\mathcal{L}^p(\Omega, \mathcal{F}, \proba)$ the space of random variables $X$ such that $\esp \left( \abs{X}^p \right) < \infty$ and by $L^p(\Omega, \mathcal{F}, \proba)$ the one identifying random variables that are equal $\proba$-a.s.
\end{note}

\subsection{Conditional calculus}

	\begin{lem}
		Let $X \in \mathcal{L}^1(\Omega, \mathcal{F}, \proba)$ and $\mathcal{G}$ a sub-$\sigma$-field of $\mathcal{F}$.
		Then there exists $Y \in \mathcal{L}^1(\Omega, \mathcal{G}, \proba)$ such that
		\begin{equation}\label{eq:condesp}
			\forall A \in \mathcal{G}, \esp(X \indic_A) = \esp(Y \indic_A)
		\end{equation}
		Moreover the following assertions hold.
		\begin{enumerate}[(i)]
			\item If $Y' \in \mathcal{L}^1(\Omega, \mathcal{G}, \proba)$ also satisfies \eqref{eq:condesp} then $Y' = Y$ $\proba$-a.s.
			\item If $X \in \mathcal{L}^2(\Omega, \mathcal{F}, \proba)$, then $Y = \proj(X \mid L^2(\Omega, \mathcal{G}, \proba))$.
			\item \eqref{eq:condesp} continues to hold extended as $\esp(XZ) = \esp(YZ)$ for all $\mathcal{G}$-measurable r.v. $Z$ such that $\esp(\abs{XZ}) < \infty$.
		\end{enumerate}
	\end{lem}

	\begin{defn}
		Let $X \in \mathcal{L}^1(\Omega, \mathcal{F}, \proba)$ and $\mathcal{G}$ a sub-$\sigma$-field of $\mathcal{F}$.
		The unique $Y \in L^1(\Omega, \mathcal{G}, \proba)$ defined by \eqref{eq:condesp} is called the \textbf{conditional expectation} of $X$ given $\mathcal{G}$, and denoted by $Y = \esp(x \mid \mathcal{G})$.
	\end{defn}

	\begin{pop}
		Suppose that $X, Y, Z, (X_n)_{n \geq 1} \in L^1(\Omega, \mathcal{F}, \proba)$.
		The following hold $\proba$-a.s.
		\begin{enumerate}[(i)]
			\item (linearity) $\forall a, b \in \R, \esp(aX + bY \mid \mathcal{G}) = a \esp(X \mid \mathcal{G}) + b \esp(Y \mid \mathcal{G})$
			\item If $X$ is $\mathcal{G}$-measurable, $\esp(X \mid \mathcal{G}) = X$
			\item If $\mathcal{G} = \{ \emptyset, \Omega \}$ is the trivial $\sigma$-field, then $\esp(X \mid \mathcal{G}) = \esp(X)$
			\item If $X$ is independent of $\mathcal{G}$ then $\esp(X \mid \mathcal{G}) = \esp(X)$
			\item (positivity) If $X \leq Y$ then $\esp(X \mid \mathcal{G}) \leq \esp(Y \mid \mathcal{G})$
			\item $\esp(X \mid \mathcal{G}) \vee \esp(Y \mid \mathcal{G}) \leq \esp(X \vee Y \mid \mathcal{G})$,
				$\esp(X \mid \mathcal{G})_+ \leq \esp(X_+ \mid \mathcal{G})$ and
				$\abs{\esp(X \mid \mathcal{G})} \leq \esp(\abs{X} \mid \mathcal{G})$
			\item (tower property) If $\mathcal{H}$ is a sub-$\sigma$-field of $\mathcal{F}$ such that $\mathcal{G} \subset \mathcal{H}$ then $\esp( \esp(X \mid \mathcal{H}) \mid \mathcal{G}) = \esp(X \mid \mathcal{G})$
			\item The expectation is not modified by conditional expectation : $\esp( \esp(X \mid \mathcal{G}) ) = \esp(X)$
			\item If $X$ is $\mathcal{G}$-measurable and $XY \in L^1(\Omega, \mathcal{F}, \proba)$, then $\esp(XY \mid \mathcal{G}) = X \cdot \esp(Y \mid \mathcal{G})$ 
		\end{enumerate} 
	\end{pop}

	\begin{defn}
		Let $Y$ be a r.v. and $\sigma(X)$ the sub-$\sigma$-field generated by a r.v. $X$.
		If $\esp(Y \mid \sigma(X))$ is well-defined, it is written as $\esp(Y \mid X)$ and is called the \textbf{conditional expectation} of $Y$ given $X$.
	\end{defn}
	
	\begin{defn}
		Let $\mathcal{G}$ be a sub-$\sigma$-field of $\mathcal{F}$.
		For any event $A \in \mathcal{F}$, we denote $\proba(A \mid \mathcal{G}) = \esp(\indic_A \mid \mathcal{G})$.
		The mapping $A \mapsto \proba(A \mid \mathcal{G})$ is called a \textbf{version of the conditional probability} of $A$ given $\mathcal{G}$.
	\end{defn}

	\begin{defn}
		Let $\mathcal{G}$ be a sub-$\sigma$-field of $\mathcal{F}$.
		A \textbf{regular version} of the conditional probability of $\proba$ given $\mathcal{G}$ is a function $\proba^{\mathcal{G}} \colon \Omega \times \mathcal{F} \to \intff{0}{1}$ such that
		\begin{enumerate}[(i)]
			\item For all $A \in \mathcal{F}$, $\proba^{\mathcal{G}}(A) \colon \omega \mapsto \proba^{\mathcal{G}}(\omega, A)$ is $\mathcal{G}$-measurable and is a version of the conditional probability of $A$ given $\mathcal{G}$, $\proba^{\mathcal{G}}(A) = \proba(A \mid \mathcal{G})$.
			\item For all $\omega \in \Omega$, tne mapping $A \mapsto \proba^{\mathcal{G}}(\omega, A)$ is a probability on $\mathcal{F}$.
		\end{enumerate}
	\end{defn}
	
	\begin{lem}
		Let $\proba^{\mathcal{G}}$ be a regular version of the conditonal probability of $\proba$ given $\mathcal{G}$ and let $Y \in L^1(\Omega, \mathcal{F}, \proba)$.
		Then $\esp(Y \mid \mathcal{G}) = \esp^{\mathcal{G}}(Y)$ $\proba$-a.s., with $\esp^{\mathcal{G}}(Y) \colon \omega \mapsto \int Y(\omega') \proba^{\mathcal{G}}(\omega, \diff \omega')$.
	\end{lem}

	\begin{defn}
		Let $\mathcal{G}$ be a sub-$\sigma$-field of $\mathcal{F}$.
		Let $(\mathsf{Y}, \mathcal{Y})$ be a measurable space and let $Y$ be an $\mathsf{Y}$-valued random variable.
		A regular version of the conditional distribution of $Y$ given $\mathcal{G}$ is a function $\proba^{Y \mid \mathcal{G}} \colon \Omega \times \mathcal{Y} \to \intff{0}{1}$ such that
		\begin{enumerate}[(i)]
			\item For all $A \in \mathcal{Y}$, $\omega \mapsto \proba^{Y \mid \mathcal{G}}(\omega, A)$ is $\mathcal{G}$ measurable and is a version of conditional distribution of $Y$ given $\mathcal{G}$, $\proba^{Y \mid \mathcal{G}}(\cdot, A) = \proba(Y \in A \mid \mathcal{G})$ $\proba$-a.s.
			\item For every $\omega$, $A \mapsto \proba^{Y \mid \mathcal{G}}(\omega, A)$ is a probability on $\mathcal{Y}$.
		\end{enumerate}
	\end{defn}

	\begin{defn}
		Let $(\mathsf{X}, \mathcal{X})$ and $(\mathsf{Y}, \mathcal{Y})$ be two measurable spaces.
		A \textbf{kernel} is a mapping $Q \colon \mathsf{X} \times \mathcal{Y} \to \intff{0}{\infty}$ satisfying the following conditions :
		\begin{enumerate}[(i)]
			\item for every $A \in \mathcal{Y}$, the mapping $Q(\cdot, A) \colon x \mapsto Q(x,A)$ is a measurable function,
			\item for every $x \in \mathsf{X}$, the mapping $Q(x, \cdot) \colon A \mapsto Q(x,A)$ is a measure on $\mathcal{Y}$.
		\end{enumerate}
		$Q$ is said to be finite if $\forall x \in \mathsf{X}, Q(x,\mathsf{Y}) < \infty$.
		It is called a probability kernel if $\forall x \in \mathsf{X}, Q(x,\mathsf{Y}) = 1$.
		It is called a Markov kernel if it is a probability kernel on $\mathsf{X} \times \mathcal{X}$.
	\end{defn}

	\begin{defn}
		Let $X$ and $Y$ be random variables with values in the measure spaces $(\mathsf{X},\mathcal{X})$ and $(\mathsf{Y},\mathcal{Y})$ respectively.
		A \textbf{regular version of the conditional distribution of $Y$ given $X$} is a probability kernel $\proba^{X \mid Y} \colon \mathsf{X} \times \mathcal{Y} \to \intff{0}{1}$ such that $\forall A \in \mathcal{Y}, \proba^{Y \mid X}(X,A) = \proba(Y \in A \mid X)$ $\proba$-a.s.
	\end{defn}

	\begin{thm}
		Let $\mathcal{G}$ be sub-$\sigma$-field of $\mathcal{F}$.
		Let $d \geq 1$ and $Y$ be an $(\R^D, \mathcal{B}(\R^d))$-valued random variable.
		Then, there exists a regular version of the conditional distribution of $Y$ given $\mathcal{G}$, $\proba^{Y \mid \mathcal{G}}$, and this version is unique in the sense that for any other regular version $\bar\proba^{Y \mid \mathcal{G}}$ of this distribution, for $\proba$-almost every $\omega$ it holds that $\forall F \in \mathcal{F}, \proba^{Y \mid \mathcal{G}}(\omega,F) = \bar\proba^{Y \mid \mathcal{G}}(\omega,F)$.
		Moreover, if $\mathcal{G} = \sigma(X)$ for some r.v. $X$ with values in a measurable space $(\mathsf{X},\mathcal{X})$, there also exists a unique regular version (hence a probability kernel) $\proba^{Y \mid X}$.
	\end{thm}

	\begin{lem}
		Let $\proba^{Y \mid X}$ bee a regular version of the conditional expectation of $Y$ given $X$.
		Then, for any real-valued measurable function $g$ on $\mathsf{Y}$ such that $\esp(\abs{g(Y}) < \infty$, we have $\esp(g(Y) \mid X) = \int g(Y) \proba^{Y \mid X}(X, \diff y)$ $\proba$-a.s.
	\end{lem}

	\begin{pop}
		
	\end{pop}


\subsection{Radon-Nikodym derivative}


\section{Transmission d'information}

	\begin{voc}
	\textbf{Causalité}
	\begin{itemize}
	\item $h$ est causale si $\forall n<0, h_{n}=0$.
	\item Un SLI est causal si sa réponse impulsionnelle est causale.
	\item Une suite $h$ est anti-causale si $\forall h \geq 0, h_n = 0$ et un SLI est anti-causal si sa RI l'est.
	\item Suite bilatère : qui n'est ni causale, ni anti-causale.
	\item RIF : un SLI à réponse impulsionnelle finie.
	\item RII : un SLI à réponse impulsionnelle infinie.
	\end{itemize}
\end{voc}

\begin{pop}
	\begin{itemize}
	\item La convolution de deux suites causale est causale.
	\item La composition de deux suites à support fini est une suite à support fini.
	\item La composition de deux SLI causaux est causale.
	\item La composition de deux SLI RIF est RIF.
	\end{itemize}
\end{pop}

\begin{defn}
	Si $h$ est un signal défini sur $\Z$ et est sommable. On appelle \textbf{transformée en $Z$} de $h$, la fonction $H$ défini $\U$, sur le cercle unité de $\C$, par
	\vspace{-0.15cm}$$
	H(z) = \sum_{n\in Z} h_{n}z^{-n}\ .
	\vspace{-0.15cm}$$
\end{defn}

\begin{pop}[Théorème d'inversion]
	Si $h$ est une suite sommable et que $H$ est sa transformée en $Z$ , alors on a :
	\vspace{-0.15cm}$$
	\forall n \in \Z, h_{n}= \int_{-1/2}^{1/2} H \left( e^{2i\pi \nu} \right) e^{2i\pi \nu n} \diff \nu\ .
	\vspace{-0.15cm}$$
	En particulier, si deux suites sommables ont la même transformées en $Z$, alors elles sont égales.
\end{pop}

\begin{pop}
	Soit $(x_{n})_n$ et $(y_{n})_n$ deux signaux sommables. On note $X$ et $Y$ leurs transformée en $Z$ et $u = x \star y$. On a :
	\vspace{-0.15cm}$$
	\forall z \in \U, U(z)=X(z)Y(z)\ .
	\vspace{-0.15cm}$$
\end{pop}

\begin{defn}[\textbf{Filtres récursifs stables}]
	Un SLI sur $\Z$ est dit récursif stable s'il vérifie les conditions suivantes :
	\begin{enumerate}
	\item
		Sa réponse impulsionnelle est sommable ($\sum_n \abs{h_n} < + \infty$).
	\item
		Il existe des coefficients $a_0,\ldots,a_p$ et $b_0,\ldots,b_q$ tels que, si $(x_n)_n$ est une entrée et $(y_n)_n$ la sortie qui lui correspond par le SLI, alors
		\vspace{-0.15cm}$$
		\forall n \in \Z, b_0 y_n + b_1 y_{n - 1} + \ldots + b_q y_{n - q} = a_0 x_n + a_1 x_{n - 1} + \ldots + a_p x_{n - p}
		\vspace{-0.15cm}$$
		Les $a_i$ et $b_j$ sont appelés coefficients du SLI.
	\item
		Les polynômes $\sum_i a_i z^i$ et $\sum_i b_i z^i$ sont premiers entre eux.
	\end{enumerate}
\end{defn}

\begin{pop}
	Si $T$ est un SLI récursif stable et $h$ sa réponse impulsionnelle, $H$ admet une transformée en $Z$ notée $H$ :
	\vspace{-0.15cm}$$
	H(z) = \frac{P(z^{-1})}{Q(z^{-1})}
	\quad \text{avec}\quad
	P = \sum_{i = 0}^p a_i X^i
	\quad \text{et}\quad
	Q = \sum_{i = 0}^q b_i X^i\ .
	\vspace{-0.15cm}$$
	En particulier $Q$ n'a pas de zéro sur $\U$.
\end{pop}

\begin{pop}
	Sous les conditions ci-dessus, pour toute suite sommable $x$, il existe une unique suite sommable $y$ qui vérifie l'équation de récurrence, donnée par $h \star x$.
\end{pop}

\begin{defn}
	On appelle \textbf{zéros} du filtre les zéros de la fonction $P(z^{-1})$, c'est à dire les inverses du polynôme $P$.
	On appelle \textbf{pôles} du filtre les zéros de $Q(z^{-1})$.
\end{defn}

\begin{pop}
	Un SLI récursif stable dont l'équation de récurrence est
	\vspace{-0.15cm}$$
	\forall n \in \Z, y_n + b_1 y_{n - 1} + \ldots + b_q y_{n - q} = a_0 x_n + a_1 x_{n - 1} + \ldots + a_p x_{n - p}
	\vspace{-0.15cm}$$
	est causal si et seulement si tous ses pôles sont dans l'intérieur du disque unité (i.e. de module strictement plus petit que $1$).
\end{pop}

\begin{defn}
	\textbf{Filtre à minimum de phase} : filtre récursif stable causal dont l'inverse est aussi stable et causal.
	Cela est équivalent à dire que ses pôles et ses zéros sont dans l'intérieur du disque unité.
\end{defn}

\begin{pop}[Implémentation des filtres récursifs]
	Soit $T$ un SLI récursif stable causal avec $b_0 = 1$, $x$ sommable, $y = T(x)$ et $x^c = (\indic_\N(n) x_n)_n$ la troncature causale de $x$.
	On considère la suite causale $t$ telle que
	\vspace{-0.15cm}$$
	\forall n \geq 0, t_n = \left( \sum_{i = 0}^p a_i x^c_{n - i} \right) - \left( \sum_{i = 1}^q a_i t_{n - i} \right)\ .
	\vspace{-0.15cm}$$
	Alors on a :
	\begin{enumerate}
	\item Si $x$ est causale alors $t = y$ (implémentation parfaite).
	\item Dans tous les cas,
		$\exists A < 1, C \geq 0, \forall n \geq 0, \abs{t_n - y_n} < C A^n \norme{x}_1$,
		i.e. pour $n$ assez grand, $t$ devient aussi proche que l'on veut de la vraie solution $y$.
	\end{enumerate}
\end{pop}

\section{Codes polaires}

	\begin{defn}
	Un \textbf{processus} est une suite $(X_n)_n$ de v.a. sur $(\Omega,\mathcal{A})$ à valeurs dans un ensemble mesuré $(E,\mathcal{E})$.
\end{defn}

\begin{defn}
	Une \textbf{filtration} de $\mathcal{A}$ est une suite croissante $\F = (\mathcal{F}_n)_{n \geq 0}$ de sous-$\sigma$-algèbres de $\mathcal{A}$.
	On dit que $(\Omega,\mathcal{A},\F)$ est un espace probabilisable filtré et $(\Omega,\mathcal{A},\F,\proba)$ un espace probabilisé filtré.
\end{defn}

\begin{ex}
	La suite $(\mathcal{F}_n^X)_{n \in \N} = (\sigma(X_i, i \leq n))_{n \in \N}$ est une filtration de $\mathcal{A}$ appelée \textbf{filtration naturelle} de $X$.
\end{ex}

\begin{defn}
	Soit $X = (X_n)_n$ un processus aléatoire et $(\mathcal{F}_n)_n$ une filtration de $\mathcal{A}$.
	On dit que $X$ est :
	\begin{itemize}
		\item[\textbullet] $\F$-\textbf{adapté} si $\forall n \in \N$, $X_n$ est $\mathcal{F}_n$-mesurable,
		\item[\textbullet] $\F$-\textbf{prévisible} si $\forall n \in \N$, $X_n$ est $\mathcal{F}_{n - 1}$-mesurable, où $\mathcal{F}_{-1} := \{ \emptyset, \Omega \}$.
	\end{itemize}
\end{defn}

\begin{defn}
	Un \textbf{temps d'arrêt} $\nu$ est une variable aléatoires à valeurs dans $\N \cup \{ \infty \}$ telle que $\forall n \in \N, \{ \nu = n \} \in \mathcal{F}_n$.
	On note $\mathcal{T}$ l'ensemble des temps d'arrêt.
\end{defn}

\begin{pop}
	Soit $(X_n)_{n \in \N}$ un processus $\F$-adapté à valeurs dans $(E,\mathcal{E})$.
	Pour tout $A \in \mathcal{E}$, on définit le \textbf{premier temps d'atteinte} $T_A := \inf \{ n \in \N \mid X_n \in A \}$, avec la convention $\inf \emptyset = \infty$.
	Alors $T_A$ est un temps d'arrêt.
\end{pop}

\begin{pop}
	Soit $\tau, \theta, (\tau_n)_{n \in \N}$ des temps d'arrêt.
	\begin{enumerate}[(i)]
		\item $\tau \wedge \theta$, $\tau \vee \theta$ et $\tau + \theta$ sont des temps d'arrêt,
		\item soit $c \geq 0$ une constante, alors $\tau + c$ et $(1 + c) \tau$ sont des temps d'arrêt,
		\item $\liminf_n \tau_n$ et $\limsup_n \tau_n$ sont des temps d'arrêt.
	\end{enumerate}
\end{pop}

\begin{pop}
	Soit $(X_n)_n$ un processus aléatoire à valeurs dans un espace mesuré $(E,\mathcal{E})$ et $\tau$ un temps d'arrêt.
	Alors $X_\tau \colon \omega \in \Omega \mapsto X_{\tau(\omega)}(\omega)$ est une v.a.
\end{pop}

\begin{pop}
	Pour tout temps d'arrêt $\tau \in \mathcal{T}$, $\mathcal{F}_\tau \subset \mathcal{A}$ est une sous-$\sigma$-algèbre de $\mathcal{A}$.
	Si $X$ est un processus aléatoire $\F$-adapté, $X_\tau$ est $\mathcal{F}_\tau$-mesurable.
\end{pop}

\begin{defn}
	L'\textbf{information disponible à un temps d'arrêt} est $\mathcal{F}_\tau := \{ A \in \mathcal{A} \mid \forall n \in \N, A \cap \{ \tau = n \} \in \mathcal{F}_n \}$.
\end{defn}

\begin{pop}
	Pour tout temps d'arrêt $\tau \in \mathcal{T}$, $\mathcal{F}_\tau$ est une sous-$\sigma$-algèbre de $\mathcal{A}$.
	Si $X$ est un processus aléatoire $\F$-adapté, $X_\tau$ est $\mathcal{F}_\tau$-mesurable.
\end{pop}

\begin{pop}
	Soit $\tau$ et $\theta$ deux temps d'arrêt.
	Alors $\{ \tau \leq \theta \}$, $\{ \tau \geq \theta \}$ et $\{ \tau = \theta \}$ appartiennent à $\mathcal{F}_\tau \cap \mathcal{F}_\theta$, et pour toute v.a. $X$ intégrable, on a $\esp(\esp(X \mid \mathcal{F}_\tau) \mid \mathcal{F}_\theta) = \esp(\esp(X \mid \mathcal{F}_\theta) \mid \mathcal{F}_\tau) = \esp(X \mid \mathcal{F}_{\tau \wedge \theta})$.
\end{pop}


\end{document}
