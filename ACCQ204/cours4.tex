\subsection{Construction d'un code cyclique}

	Code cyclique primitif : de longueur $n = p^m - 1$ avec $p$ premier.
	
	\begin{thm}
		Soit $\beta_1,\ldots,\beta_r$ les racines dans $\F_{p^m}$ de $g(X)$, polynôme générateur d'un code cyclique primitif.
		Alors $c(X) \in \F_p[X]$ est un mot de code si et seulement si $\forall i, c(\beta_i) = 0$.
		De plus $g(X) = \ppcm(f_{\beta_1}(X),\ldots,f_{\beta_r}(X))$ où les $f_i$ sont les polynômes minimaux.
	\end{thm}

	On obtient $g(X)$ en choisissant les racines.
	On a alors $k$ via $\deg(g(X)) = n - k$.
	
	Les zéros du code sont les $i$ tels que $\alpha^i$ est racine de $g(X)$, avec $\alpha$ un élément primitif.


\subsection{Les codes BCH}

	\begin{defn}
		\textbf{Code BCH} : code cyclique ayant $2t$ zéros consécutifs (qui corrige $t$ erreurs, $d_{\min} \geq 2t + 1$).
	\end{defn}
	
	Construction d'un code BCH, avec $n = p^m - 1$, $p$ premier :
	\begin{enumerate}[1)]
		\item choisir $p(X)$ premier, de degré $m$ sur $\F_p[X] \longrightarrow \F_{p^m}$,
		\item calculer les polynômes minimaux des $\alpha^i$ pour $1 \leq i \leq 2t$,
		\item calculer $g(X) = \ppcm((f_{\alpha^i})_{i = 1,\ldots,2t})$.
	\end{enumerate}


\subsection{Décodage par calcul du syndrôme}

	On a $R(X) = c(X) + e(X)$.
	Le polynôme syndrôme $s(X)$ est le reste de $R(X)/g(x)$.
	
	\begin{defn}
		\textbf{vecteur syndrôme} : $S = (s_1,\ldots,s_{2t})$ où $s_i = R(\alpha^i)$.
	\end{defn}

	Si $S = O$, $R(x)$ est un mot de code.
	On a $s_{2i} = R(\alpha^{2i}) = R(\alpha^i)^2 = s_i^2$, donc il y a de la redondance.

	\begin{ex}[Code de Hamming]
		$BCH(2^m - 1, 2^m - 1 - m, 3)$.
		On a $\GF(2^m) = \GF(2)[X]/P(X)$, $g(X) = p(X)$.
		Le code corrige une erreur.
		Algorithme de décodage : calculer $s_1$, si $s_1 = 0$, $e(x) = 0$, sinon $s_1 = \alpha^i \neq 0$ et $e(X) = X^i$.
	\end{ex}

	\begin{ex}[Code BCH binaire correcteur de 2 erreurs]
		$S = (s_1,s_2,s_3,s_4)$, $s_2 = s_1^2$ et $s_4 = s_1^4$.
		Les composantes non redondantes sont $s_1$ et $s_3$.
		\begin{tabular}{cccc}
			$e(X) = 0$ & pas d'erreur & $s_1 = 0$ & $s_3 = 0$ \\
			$e(X) = X^i$ & erreur en position $i$ & $s_1 = \alpha^i$ & $s_3 = \alpha^{3i}$ \\
			$e(X) = X^i + X^j$ & erreurs en positions $i$ et $j$ & $s_1 = \alpha^i + \alpha^j$ & $s_3 = \alpha^{3i} + \alpha^{3j}$
		\end{tabular}
	\end{ex}

	\begin{defn}
		\textbf{Polynôme localisateur d'erreurs} : $\Lambda(X)$ dont les racines sont les inverses des positions des erreurs dans $\GF(p^m)[X]$ (corps localisateur d'erreurs).
	\end{defn}

	Dans l'exemple : $\Lambda(X) = (1 + \alpha^i X)(1 + \alpha^j X) \in \GF(2^m)[X]$ avec 2 erreurs.
	En développant on obtient $\Lambda(X) = 1 + s_1 X + \left( s_1^2 + \frac{s_3}{s_4} \right) X^2$.


\subsection{Transformée de Fourier discrète dans les corps finis}

	\begin{itemize}
		\item[\textbullet] TDF dans $\C^n$ : $(h_0,\ldots,h_{n - 1}) \mapsto (H_0,\ldots,H_{n - 1})$ avec $H_k = \sum_{l = 0}^{n - 1} h_l \exp \left( - \frac{2i\pi k}{n} l \right)$.
		\item[\textbullet] TDF dans $\GF(p^m)$ : soit $\alpha \in \GF(p^m)$ tel que $\alpha^n = 1$, on a
			$$(v_0,\ldots,v_{n - 1}) \in \GF(p) \longleftrightarrow (V_0,\ldots,V_{n - 1}) \in \GF(p^m)$$
			$$v_i = \frac{1}{n \mod{p}} \sum_{j = 0}^{n - 1} V_j \alpha^{-ij} \qquad V_j = \sum_{i = 0}^{n - 1} v_i \alpha^{ij}$$
	\end{itemize}

	Dans le cas $p = 2$ et $n = 2^m - 1$, on a donc $v_i = \sum_{j = 0}^{n - 1} V_j \alpha^{-ij}$ et $V_j = \sum_{i = 0}^{n - 1} v_i \alpha^{ij}$.
	
	On définit $v(X) = \sum_{i = 0}^{n - 1} v_i X^i$ et $V(X) = \sum_{j = 0}^{n - 1} V_j X^j$.
	On a $v_i = V(\alpha^{-i})$ et $V_j = v(\alpha^j)$.

	\paragraph{Produit de convolution cyclique :}
	$$v_i = h_i u_i \overset{\text{TDF}}{\longrightarrow} V_j = \sum_{i = 0}^{n - 1} H_i U_{(j - i) \mod{n}}$$
	$$v_i = \sum_{j = 0}^{n - 1} h_j u_{(j - i) \mod{n}} \overset{\text{TDF}}{\longrightarrow} V_j = H_j U_j$$


\subsection{Technique spectrale de décodage des codes BCH}

	Vecteur reçu : $v(X) = c(X) + e(X)$ ($v_i = c_i + e_i$).
	Syndromes : $\forall i \in \intiff{1}{2t}, S_i = v(\alpha^i) = e(\alpha^i) = E_i$.

	$$\underset{e_0,\ldots,e_{n - 1}}{e(X)} \overset{\text{TDF}}{\longrightarrow} \underset{E_0,\ldots,E_{n - 1}}{E(X)}$$
	Or $(E_1,\ldots,E_{2t}) = (S_1,\ldots,S_{2t})$ est la fenêtre spatiale sur le motif d'erreur.

	Supposons $\nu$ erreur, avec $\nu \leq t$, de positions $i_k, k \in \intiff{1}{\nu}$.

	On a le polynôme localisateur d'erreurs $\Lambda(X) = \prod_{k = 1}^{\nu} (1 + \alpha^{i_k} X) = \sum_{i = 0}^{n - 1} \Lambda_i X^i$.
	On passe à $\lambda(X) = \sum_{i = 0}^{n - 1} \lambda_i X^i$, $\lambda_i = \Lambda(\alpha^i) = \sum_{j = 0}^{n - 1} \Lambda_j \alpha^{-ij} \in \GF(p)$.
	
	\begin{pop}
		On a $\forall i, e_i \lambda_i = 0$.
	\end{pop}

	\begin{proof}
		Si $i$ n'est pas la position d'une erreur, $e_i = 0$, sinon $i$ est la position d'une erreur, donc $e_i \neq 0$ mais $\lambda_i = \Lambda(\alpha^{-1}) = 0$.
	\end{proof}

	Système fondamental de décodage, $t$ équations avec $t$ inconnues :
	$$\left\{ \begin{array}{l}
		\sum_{j = 0}^{n - 1} \Lambda_j E_{(k - j) \mod{n}} = 0 \\
		\forall i \in \intiff{t + 1}{2t}, \sum_{j = 1}{t} \Lambda_j S_{k - j} = S_k
		\end{array} \right.$$

	\paragraph{Algorithme de PGZ.} Ériture du système sous forme matricielle :
	$$\underset{\text{matrice des syndromes}}{\underbrace{\begin{pmatrix}
		S_1    & S_2       & \ldots & S_t \\
		S_2    & S_3       & \ldots & S_{t + 1} \\
		\hdots & \hdots    &        & \hdots \\
		S_t    & S_{t + 1} & \ldots & S_{2t}
		\end{pmatrix}}} \cdot
		\begin{pmatrix} \Lambda_t \\ \Lambda_{t - 1} \\ \hdots \\ \Lambda_1 \end{pmatrix}
		=
		\begin{pmatrix} S_{t + 1} \\ S_{t + 2} \\ \hdots \\ S_{2t} \end{pmatrix}$$

	S'il existe $t$ erreurs, la matrice des syndromes est inversible, donc le système est résoluble et il existe une solution unique.
	Sinon on reprend mais en testant pour une erreur de moins.
