Les codes convolutifs ont été inventés par Elias en 1954 et sont différents des codes en blocs.

\subsection{Principe}

	Il y a un effet mémoire : la sortie d’un codeur convolutif dépend d’un symbole courant à coder ainsi que du symbole précédent et du résultat de codage du symbole précédent.

	Un \textbf{codeur convolutif} est une machine à états finis.

	Le code possède un \textbf{rendement} $R = \frac{k}{n}$ où $k$ est le nombre de symboles en entrée et $n$ le nombre de symoboles en sortie.
	Il a également une \textbf{longueur de contrainte} qui correspond au nombre de temps horloge influençant les sorties du codeur.
	Pour un 1/2-taux codeur, $L = n + 1$.

	\begin{ex}
		Soit un codeur avec $R = \frac{1}{2}$ et $L = 3$.
		
		% TODO
	\end{ex}

\subsection{Représentation polynomiale}

	Soit une séquence binaire $e_i(1),\ldots,e_i(n)$ sur l'entrée $i$.
	On lui associe le polynôme $e_i(X) = \sum_{t = 1}^n e_i(t) X^t$.
	Alors chaque sortie possède une représentation polynomiale comme $s_i(X) = \sum_{j = 1}^k g_{ij}(X) e_j(X)$.
	
	Les $g_{ij}$ sont appelés les \textbf{polynômes générateurs} du codes.

	\begin{ex}
		Sur l'exemple précédent on a $s_1^{(i)} = e^{(i)} + e^{(i - 1)} + e^{(i - 2)}$ et $s_2^{(i)} = e^{(i)} + e^{(i - 2)}$.
		Donc $s_1(X) = e(X) + X e(X) + X^2 e(X) = g_1(X) e(X)$ avec $g_1(X) = 1 + X + X^2$ et $s_2(X) = e(X) + X^2 e(X) = g_2(X) e(X)$ avec $g_2(X) = 1 + X^2$.
	\end{ex}
	
	\begin{ex}
		Prenons le code avec $R = \frac{2}{3}$ et $L = 2$ donné par le circuit suivant :
		% TODO
		
		On a alors :
		$$s_1(X) = \underset{g_{11}(X)}{\underbrace{(1 + X)}} \cdot e_1(X) + \underset{g_{12}(X)}{\underbrace{X}} \cdot e_2(X)$$
		$$s_2(X) = \underset{g_{21}(X)}{\underbrace{1}} \cdot e_1(X) + \underset{g_{22}(X)}{\underbrace{(1 + X)}} \cdot e_2(X)$$
		$$s_3(X) = \underset{g_{31}(X)}{\underbrace{(1 + X)}} \cdot e_1(X) + \underset{g_{32}(X)}{\underbrace{0}} \cdot e_2(X)$$
		Donc $G = \begin{pmatrix}
			1 + X & 1 & 1 + X \\
			X & 1 + X & 0
			\end{pmatrix} \in \M_{k,n}$
		est la matrice génératrice du code.
	\end{ex}

	\begin{defn}
		\textbf{Code convolutif récursif} : une partie de la sortie est réintroduite dans les étages mémoire.
		Les polynômes signatures sont remplacés par des quotients de polynômes.
	\end{defn}
	
	\begin{defn}
		\textbf{Code convolutif systématique} : une partie de ses sorties est exactement égale à ses entrées.
	\end{defn}

	\begin{ex}
		Sur l'exemple 1 : on crée un codeur systématique.
		On passe de $e(X) \to \left\{ \begin{array}{l}
			s_1(X) = (1 + X + X^2) e(X) \\
			s_2(X) = (1 + X^2) e(X)
			\end{array} \right.$
		à $e(X) \to \left\{ \begin{array}{l}
			f_1(X) = e(X) \\
			f_2(X) = \frac{1 + X^2}{1 + X + X^2} e(X)
			\end{array} \right.$.
		Donc $f_2(X) \cdot (1 + X + X^2) = (1 + X^2) e(X)$, d'où
		$$f_2(X) = e(X) + X^2 e(X) + X f_2(X) + X^2 f_2(X) = e(X) + X f_2(X) + X^2 (e(X) + f(X))\ .$$
		
		% TODO
	\end{ex}
	
	\begin{ex}
		Sur l'exemple 2 avec $R = \frac{2}{3}$.
		On prend les deux premières colonnes de $G$ pour former $G_1 = \begin{pmatrix} 1 + X & 0 \\ 1 & 1 + X \end{pmatrix}$.
		La nouvelle matrice est alors
		$$G' = G_1^{-1} \cdot G
			= \begin{pmatrix} 1 & 0 & p_1(X)/p_2(X) \\ 0 & 1 & p_3(X)/p_1(X) \end{pmatrix}
			= \begin{pmatrix} 1 & 0 & \frac{1}{1 + X} \\ 0 & 1 & \frac{1 + X + X^2}{1 + X^2} \end{pmatrix}$$
		
		% TODO
		
		$$f_3(X) = \frac{1}{1 + X} e_1(X) + \frac{1 + X + X^2}{1 + X} e_2(X)$$
		$$(1 + X^2) f_3(X) = \frac{1 + X^2}{1 + X} e_1(X) + (1 + X + X^2) e_2(X)$$
		\begin{align*}
		f_3(X)
			& = X^2 f_3(X) + e_1(X) + X e_1(X) + e_2(X) + X e_2(X) + X^2 e_2(X) \\
			& = e_1(X) + e_2(X) + X(e_1(X) + e_2(X)) + X^2 (f_3(X) + e_2(X))
		\end{align*}
	\end{ex}


\subsection{Codes catastroophiques}

	\begin{ex}
		Exemple avec $R = \frac{1}{2}$, $L = 3$ :
		
		% TODO
		
		$1111 \ldots 1 \longrightarrow 11010000 \ldots 0$.
	\end{ex}
	
	\begin{defn}
		Un code est dit \textbf{catastrophique} s'il existe une séquence en entrée de poids infini qui donne une séquence en sortie de poids fini.
	\end{defn}

	\begin{pop}
		Pour un code de rendement $R = \frac{1}{n}$, le code est catastrophique ssi $\pgcd(g_i(X)) \neq 1$.
		Pour un code de rendement $R = \frac{k}{n}$, le code est catastrophique ssi le pgcd des mineurs d'ordre $k$ de la matrice $G$ est différent de $1$.
	\end{pop}


\subsection{Représentations graphiques}

	% TODO
	

\subsection{Algorithme de décodage}

	Décodage optimal ML : on suppose les séquences d'entrées équiprobables.
	Pour un BSC on aurait $ML = \min_{c \in C} d_H(r,c)$ avec $r$ le message reçu.
	
	% TODO

\subsection{Distance libre et fonction de transfert}

	% TODO

\subsection{Poinçonnage des codes convolutifs}

	% TODO
