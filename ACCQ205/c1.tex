\subsection{Anneaux, algèbres, corps, idéaux premiers et maximaux et corps des fractions}

	On considère les anneaux commutatifs sauf précision contraire.

	\begin{defn}
		Soit $k$ un anneau.
		Une $k$-\textbf{algèbre} (commutative) est un anneau $A$ muni d'un morphisme d'anneaux $\varphi_A \colon k \to A$, appelé \textbf{morphisme structural} de l'algèbre, dont l'image est contenue dans le centre de $A$.
	\end{defn}
	
	Formellement une $k$-algèbre est le couple $(A,\varphi_A)$ mais on le réduit souvent à la donnée de $\varphi_A$.
	De façon équivalente une $k$-algèbre est un $k$-module qui est muni d'une multiplication $k$-bilinéaire qui en fait un anneau.
	
	\begin{defn}
		Un \textbf{morphisme de $k$-algèbres} est un morphisme d'anneaux $\psi \colon A \to B$ tel que $\varphi_B = \psi \circ \varphi_A$.
		Ce sont aussi les applications $k$ linéaires qui préservent la multiplication.
	\end{defn}

	\begin{rem}
		Une $\Z$-algèbre est exactement la même chose qu'un anneau.
	\end{rem}

	En pratique $k$ est généralement un corps et $A$ est donc un $k$-ev muni d'une multiplication $k$-bilinéaire qui en fait un anneau.
	
	\begin{defn}
		Un élément $a$ d'un anneau $A$ est dit \textbf{régulier} si $x \mapsto ax$ est injectif, i.e. $ax = 0 \implies x = 0$ (il est inversible si bijectivité de l'application).
		$A$ est dit \textbf{intègre} si tous les éléments sauf $0$ sont réguliers, i.e. $0 \neq 1$ et $\forall a,b \in A \setminus \{ 0 \}, ab = 0 \implies a = 0 \text{ ou } b = 0$.
		Par convention l'anneau nul n'est pas intègre.
	\end{defn}

	\begin{defn}
		Un idéal $\mathfrak{p}$ d'un anneau $A$ est dit \textbf{premier} lorsque l'anneau quotient $A / \mathfrak{p}$ est intègre, i.e. $\mathfrak{p} \neq A$ et $\forall a,b \in A, ab \in \mathfrak{p} \implies a \in \mathfrak{p} \text{ ou } b \in \mathfrak{p}$.
	\end{defn}

	\begin{pop}
		Dans un anneau $A$, l'ensemble $A^\times$ des inversibles est un groupe, aussi appelé groupe des \textbf{unités de $A$}.
	\end{pop}
	
	\begin{defn}
		Un \textbf{corps} est un anneau $k$ dans lequel $k^\times = k \setminus \{ 0 \}$.
		C'est équivalent à dire que $k$ a deux idéaux qui sont $\{ 0 \}$ et lui-même.
		C'est en particulier un anneau intègre.
		Par convention l'anneau nul n'est pas un corps.
	\end{defn}

	\begin{defn}
		Un idéal $\mathfrak{m}$ d'un anneau $A$ est dit \textbf{maximal} si $A / \mathfrak{m}$ est un corps.
		De façon équivalente $\mathfrak{m} \neq A$ et $\mathfrak{m}$ est maximal pour l'inclusion parmi les idéaux différents de $A$.
	\end{defn}

	\begin{pop}
		Un idéal maximal est premier.
	\end{pop}

	\begin{ex}
		Dans un anneau factoriel $A$, un idéal de la forme $(f)$ avec $f \in A$ est premier ssi $f$ est nul ou irréductible.
	\end{ex}

	\begin{lem}[Principe maximal de \textbf{Hausdorff}]
		Soit $\mathcal{F} \subset \mathcal{P}(A)$ non vide et tel que, pour tout partie $\mathcal{I} \subset \mathcal{F}$ non vide totalement ordonnée par l'inclusion, $\exists F \in \mathcal{F}, \bigcup_{i \in \mathcal{I}} I \subset F$.
		Alors il existe $M \in \mathcal{F}$ maximal pour l'inclusion.
	\end{lem}

	\begin{pop}
		Dans un anneau $A$, tout idéal strict (autre que $A$) est inclus dans un idéal maximal.
	\end{pop}
	
	\begin{defn}
		Un élément $x$ d'un anneau $A$ est dit \textbf{nilpotent} lorsque $\exists n \in \N, x^n = 0$.
		Si $0$ est le seul élément nilpotent, $A$ est dit réduit.
	\end{defn}
	
	\begin{pop}
		Dans un anneau, l'ensemble des éléments nilpotents est un idéal appelé \textbf{nilradical} de l'anneau.
		C'est aussi l'intersection des idéaux premiers de l'anneaux.
		Le quotient de l'anneau par son nilradical est réduit.
	\end{pop}
	
	\begin{defn}
		Soit $A$ un anneau intègre.
		On définit le \textbf{corps des fractions} de $A$, $\Frac(A) = \left\{ \frac{a}{q} \mid a \in A, q \in A \setminus \{ 0 \} \right\}$ en convenant d'identifier $\frac{a}{q}$ avec $\frac{a'}{q'}$ lorsque $aq' = a'q$.
	\end{defn}

	\begin{pop}
		Soit $A$ un anneau intègre, $K$ un corps et $\varphi \colon A \to K$ un morphisme d'anneau injectif.
		Alors il existe un unique morphisme de corps $\hat{\varphi} \colon \Frac(A) \to K$ qui prolonge $\varphi$ et il est donné par $\hat{\varphi} \left( \frac{a}{q} \right) = \frac{\varphi(a)}{\varphi(q)}$.
	\end{pop}

	\begin{defn}
		Le corps des fractions de l'anneau des polynômes $k[t_1,\ldots,t_n]$ est appelé corps des \textbf{fractions rationnelles} et noté $k(t_1,\ldots,t_n)$.
	\end{defn}

	\begin{pop}
		Soit $k$ un corps et $K$ une $k$-algèbre de dimension finie intègre.
		Alors $K$ est un corps.
	\end{pop}

	\begin{lem}[de \textbf{Gauß}]
		Soit $A$ un anneau factoriel et $K$ son corps des fractions.
		Alors :
		\begin{enumerate}[(i)]
			\item $A[t]$ est factoriel,
			\item $f \in A[t]$ est irréductible ssi $f$ est constant et irréductible dans $A$, ou bien $f$ est primitif, i.e. irréductible dans $K[t]$ et le pgcd dans $A$ de ses coefficients vaut 1.
		\end{enumerate}
	\end{lem}


\subsection{Algèbres engendrée, extensions de corps}

	\begin{defn}
		Soit $A$ une $k$-algèbre et $(x_i)_{i \in I}$ famille de $A$.
		La \textbf{$k$-algèbre engendrée} $k[x_i]_{i \in I}$ dans $A$ par les $x_i$ est l'intersection de toutes les sous-$k$-algèbres de $A$ contenant les $x_i$.
		C'est la plus petite sous-$k$-algèbre contenant les $x_i$.
		Elle est dite \textbf{de type fini} si $I$ est fini.
	\end{defn}

	On peut aussi décrire $k[x_i]_{i \in I}$ comme l'ensemble de tous les éléments de $A$ qui peuvent être obtenus à partir de $1$ et des $x_i$ par les opérations $\times$, $\cdot$ et $+$.
	
	\begin{defn}
		Une \textbf{extension de corps} est un morphisme d'anneaux $k \to K$ entre corps ($K$ est une $k$-algèbre qui est un corps).
		On note $k \subseteq K$ ou $K/k$ et l'on dit que $k$ est un \textbf{sous-corps} de $K$.
	\end{defn}
	
	\begin{defn}
		Soit $k \subseteq K$ une extension de corps et $(x_i)_{i \in I}$ une famille de $K$.
		La \textbf{sous-extension engendrée} (dans $K$) par les $x_i$, notée $k(x_i)_{i \in I}$, est l'intersection de tous les sous-corps de $K$ contenant $k$ et les $x_i$.
		C'est le plus petit corps intermédiaire contenant les $x_i$.
		Elle est dite \textbf{de type fini} si $I$ est fini.
	\end{defn}
	
	Ce sont les valeurs des fractions rationnelles à coefficients dans $k$ évaluées en des $x_i$.
	
	\begin{pop}
		Une sous-extension d'une extension de corps de type fini est de type fini.
		Mais une sous-algèbre d'une algèbre de type fini n'est pas, en général, de type fini !
	\end{pop}


\subsection{Extension algébrique et degré}

	\begin{defn}
		Soit $k \subseteq K$ est une extension et $x \in K$.
		L'extension $k \subseteq k(x)$ est dite \textbf{monogène}.
	\end{defn}
	
	Avec ce $x$, on définit $\varphi \colon \begin{array}{rcl} k[t] & \to & K \\ P & \mapsto & P(x) \end{array}$ le morphisme d'évaluation, le seul à envoyer l'indéterminée $t$ sur $x$.
	Alors $\Ker(\varphi)$ est un idéal de $k[t]$ et l'on est dans l'un des deux cas suivants :
	\begin{itemize}
		\item[\textbullet] $\varphi$ est injectif, $x$ est \textbf{transcendant} sur $k$, $\varphi$ se prolonge de manière unique en une extension de corps $k(t) \to K$, et l'image de $k(t)$ (corps des fractions rationnelles) est $k(x)$ (extension).
		\item[\textbullet] ou $\Ker(\varphi)$ est engendré par $\mu_x \in k[t]$ unitaire, appelé \textbf{polynôme minimal} de $x$ et $x$ est dit \textbf{algébrique}. Alors $\varphi(k[t])$ s'identifie à la $k$-algèbre $k[t]/(\mu_x)$ de dimension $\deg(\mu_x)$, appelé \textbf{degré} de $x$.
	\end{itemize}
	
	\begin{rem}
		Les algébriques de degré 1 sur $k$ sont exactement les éléments de $k$.
	\end{rem}
	
	\begin{rem}
		Si $k \subseteq k' \subseteq K$, le polynôme minimal d'un $x \in K$ sur $k'$ divise celui sur $k$.
	\end{rem}
	
	\begin{defn}
		Soit $\mu \in k[t]$ unitaire irréductible.
		Le \textbf{corps de rupture} de $\mu$ sur $k$ est $k[t]/(\mu)$.
		%Il coïncide avec $k$ ssi $\deg(\mu) = 1$.
	\end{defn}
	
	\begin{defn}
		Une extension de corps $k \subseteq K$ est dite \textbf{algébrique} (« au-dessus » de $k$, ou « sur » $k$) lorsque chaque élément de $K$ est algébrique sur $k$.
	\end{defn}
	
	\begin{defn}
		Un corps $k$ est \textbf{algébriquement clos} si sa seule extension algébrique est lui-même.
		Cela revient à dire que les seuls polynômes unitaires irréductibles dans  $k[t]$ sont les $t - a$.
	\end{defn}
	
	\begin{defn}
		Soit $k \subseteq K$ une extension de corps.
		Considérant $K$ comme un $k$-ev, sa dimension (finie ou infinie) est notée $[K:k]$ et appelée \textbf{degré de l'extension}.
		Une extension de degré fini est dite \textbf{finie}.
	\end{defn}
	
	\begin{pop}
		L'extension monogène $k \subseteq k(x)$ est finie si et seulement si $x$ est algébrique sur $k$, et dans ce cas $k(x) \simeq k[t]/(\mu_x)$ et $[k(x):k] = \deg(\mu_x) = \deg(x)$.
	\end{pop}
	
	\begin{pop}
		Soit $k \subseteq K \subseteq L$ deux extensions imbriquées.
		Alors $[L:k] = [K:k] [L:K]$.
	\end{pop}
	
	\begin{cor}
		\begin{itemize}
			\item[\textbullet] Une extension $k \subseteq k(x_1,\ldots,x_n), n \in \N$ avec $x_1,\ldots,x_n$ algébriques est finie et a une base comme $k$-ev formée de monômes en les $x_1,\ldots,x_n$ (i.e. de la forme $x_1^{r_1} \cdots x_n^{r_n}$).
			\item[\textbullet] Une extension est finie si et seulement si elle est à la fois algébrique et de type fini.
			\item[\textbullet] Une extension de corps engendrée par une famille quelconque d'éléments algébriques est algébrique.
				Donc les sommes, différences, produits et inverses de quantités algébriques sur $k$ le sont aussi.
			\item[\textbullet] Si $k \subseteq K$ et $K \subseteq L$ sont algébriques alors $k \subseteq L$ l'est.
		\end{itemize}
	\end{cor}
	
	\begin{defn}
		Soit $k \subseteq K$ une extension de corps.
		Le corps des éléments de $K$ algébriques sur $k$ est appelé \textbf{fermeture algébrique} de $k$ dans $K$.
		Si c'est précisément $k$, on dit que $k$ est \textbf{algébriquement fermé} dans $K$.
	\end{defn}
	
	\begin{pop}
		Un corps algébriquement clos est algébriquement fermé dans toute extension (mais pas l'inverse en général).
	\end{pop}
	
	\begin{rem}
		Soit $K$ algébrique au-dessus de $k$ et $t_1,\ldots,t_n$ des indéterminées.
		Alors $K(t_1,\ldots,t_n)$ est algébrique sur $k(t_1,\ldots,t_n)$.
	\end{rem}


\subsection{Extensions linéairement disjointes}

	\begin{defn}
		Soit $k \subseteq K$ et $k \subseteq L$ deux extensions contenues dans une même troisième $M$.
		On dit qu'elles sont \textbf{linéairement disjointes} lorsque toute famille d'éléments de $K$ linéairement indépendante sur $K$ est encore linéairement indépendantes sur $L$ en tant que famille d'éléments de $M$.
	\end{defn}
	
	\begin{rem}
		Cette condition est symétrique et l'on a $K \cap L = k$.
		On appelle \textbf{composé} de $K$ et $L$ le sous-corps de $M$ engendré par $K$ et $L$ : $K.L = k(K \cup L) = K(L) = L(K)$.
	\end{rem}
	
	\begin{pop}
		Soit $k \subseteq K$ et $k \subseteq L$ deux extensions contenues dans une troisième $M$ et $(v_j)$ une base de $K$ comme $k$-ev.
		Alors $K$ et $L$ sont linéairement disjoints si et seulement si $(v_i)$ est encore linéairement indépendante sur $L$ quand on la voit comme une famille d'éléments de $M$.
	\end{pop}
	
	\begin{pop}
		Soit $k \subseteq K$ $k \subseteq L$ deux extensions, l'une algébrique, contenues dans $M$.
		Alors $K.L$ est le sous-$k$-ev $\Vect(\{ xy, x \in K, y \in L \} )$ de $M$ et toute base de $K$ sur $k$ est encore une base de $K.L$ sur $L$.
	\end{pop}
	
	\begin{cor}
		On a $[K.L : L] = [K : k]$ et $[K.L : k] = [K : k] \cdot [L : k]$.
	\end{cor}
	
	\begin{pop}
		Soit $k \subseteq K$ une extension de corps et $t_1,\ldots,t_n$ des indéterminées.
		Alors les extensions $k \subseteq K$ et $k \subseteq k(t_1,\ldots,t_n)$ sont linéairement disjointes dans $K(t_1,\ldots,t_n)$.
		Si de plus $K$ est algébrique sur $k$, alors toute base de $K$ comme $k$-ev est une base de $K(t_1,\ldots,t_n)$ comme $k(t_1,\ldots,t_n)$-ev.
	\end{pop}


\subsection{Bases et degré de transcendance}

	\begin{defn}
		Soit $k \subseteq K$ une extension de corps.
		Une famille finie $x_1,\ldots,x_n \in K$ est dite \textbf{algébriquement indépendante} sur $k$ lorsque le seul polynôme $P \in k[t_1,\ldots,t_n]$ tel que $P(x_1,\ldots,x_n) = 0$ est le polynôme nul.
		En particulier, chacun des $x_i$ est transcendant sur $k$, et un unique $x \in K$ est algébriquement indépendant sur $k$ si et seulement s'il est transcendant sur $k$.
		Une famille infinie est algébriquement indépendante si toute sous-famille finie l'est.
	\end{defn}
	
	\begin{defn}
		\textbf{Base de transcendance} : famille $(x_i)_{i \in I}$ de $K$ algébriquement indépendante sur $k$ telle que $K$ est algébrique au-dessus de l'extension $k(x_i)_{i \in I}$.
	\end{defn}
	
	\begin{rem}
		Des indéterminées $t_1,\ldots,t_n$ sont algébriquement indépendantes et si $x_1,\ldots,x_n$ sont algébriquement indépendants alors $k(x_1,\ldots,x_n)$ s'identifie à $k(t_1,\ldots,t_n)$ et l'extension $k \subseteq k(x_1,\ldots,x_n)$ est dite \textbf{transcendante pure}.
	\end{rem}
	
	\begin{pop}
		Soit $k \subseteq K$ une extension de corps. On a :
		\begin{itemize}
			\item[\textbullet] Toute famille de $K$ algébriquement indépendante sur $k$ se complète en une base de transcendance de $K$ sur $k$.
			\item[\textbullet] De toute famille qui engendre $K$ en tant qu'extension de corps de $k$, ou même qui engendre un corps intermédiaire $E$ au-dessus duquel $K$ est algébrique, on peut extraire une base de transcendance.
			\item[\textbullet] (lemme d'échange) Soit $z_1,\ldots,z_n$ une base de transcendance finie de $K$ sur $k$ et $t \in K$ tel que $z_1,\ldots,z_l,t$ soit algébriquement indépendants sur $k$ pour un certain $l \geq 0$.
				Alors $\exists j \in \iniff{l + 1}{n}$ tel qu'en remplaçant $z_j$ par $t$ dans $z_1,\ldots,z_n$ on obtienne encore une base de transcendance.
			\item[\textbullet] Deux bases de transcendance de $K$ sur $k$ ont toujours le même cardinal.
		\end{itemize}
	\end{pop}
	
	\begin{defn}
		Soit $k \subseteq K$ une extension.
		Le cardinal d'une base de transcendance de $K$ sur $k$ est le \textbf{degré de transcendance} de $K$ sur $k$, noté $\degtr_k(K)$ (nul ssi l'extension est algébrique).
	\end{defn}
	
	\begin{pop}
		Soit $k \subseteq  K \subseteq L$ une tour d'extensions.
		Alors $\degtr_k(L) = \degtr_k(K) + \degtr_K(L)$.
	\end{pop}
	
	\begin{pop}
		Soit $k \subseteq  k' \subseteq K$ une tour d'extensions avec $k'$ algébrique sur $k$.
		Alors si $(x_i)_{i \in I}$ est une famille de $K$ algébriquement indépendants sur $k$, ils le sont encore sur $k'$.
		De plus, toute base de $k'$ comme $k$-ev est encore une base de $k'(x_i)_{i \in I}$ sur $k(x_i)_{i \in I}$, et $[k'(x_i)_{i \in I} : k(x_i)_{i \in I}] = [k' : k]$.
	\end{pop}



\subsection{Corps de rupture, corps de décomposition et clôture algébrique}

	\begin{defn}
		Soit $K$ un corps et $\mu \in K[t]$ irréductible.
		On appelle \textbf{corps de rupture} de $\mu$ sur $K$ une extension $K \subseteq L$ telle que $\mu$ admette une racine $x \in K$ pour laquelle $L = K(x)$.
	\end{defn}

	\begin{pop}
		Soit $K$ un corps et $\mu \in K[t]$ irréductible. Alors :
		\begin{enumerate}[(i)]
			\item Il existe un corps de rupture de $\mu$ sur $K$, à savoir $K[t]/(\mu)$.
			\item Soit $K \subseteq L$ un corps de rupture de $\mu$ sur $K$ avec $L = K(x)$ et $K \subseteq L'$ une extension dans laquelle $\mu$ a une racine $x'$.
				Alors il existe un unique morphisme de corps $\varphi \colon L \to L'$ tel que $\varphi_{|K} = \Id_K$ et $\varphi(x) = x'$.
			\item Si, en plus de \textit{(ii)}, $K \subseteq L'$ est aussi un corps de rupture de $\mu$ sur $K$, $\varphi$ est un isomorphisme.
		\end{enumerate}
	\end{pop}

	\begin{defn}
		Soit $K$ un corps et $f \in K[t]$.
		On appelle \textbf{corps de décomposition} de $f$ sur $K$ une extension $K \subseteq L$ telle que $f$ soit scindé sur $L$.
		Pour une famille $(f_i)$ de polynômes, il s'agit d'une extension de $K$ dans laquelle tous les $f_i$ sont scindés, et qui est engendrée en tant que corps par l'ensemble de toutes les racines de tous les $f_i$.
	\end{defn}

	%\begin{pop}
		%Soit $K$ un corps et $f \in K[t]$. Alors :
		%\begin{enumerate}[(i)]
			%\item Il existe un corps de décomposition de $f$ sur $K$.
			%\item Soit $K \subseteq L$ un corps de décomposition de $f$ sur $K$ et $K \subseteq L'$ une extension dans laquelle $f$ est scindé.
				%Alors il existe un unique morphisme de corps $\psi \colon L \to L'$ tel que $\psi_{|K} = \Id_K$.
			%\item Si $f$ est irréductible et $x,x'$ racines de $f$ dans $L$ et $L'$ respectivement, on peut choisir $\psi$ tel que $\psi(x) = x'$.
			%\item Si $K \subseteq L'$ est aussi un corps de décomposition de $f$ sur $K$, tout $\psi$ comme en \textit{(ii)} est un isomorphisme.
		%\end{enumerate}
	%\end{pop}

	\begin{pop}
		Soit $K$ un corps et $(f_i)$ une famille quelconque de $K[t]$. Alors :
		\begin{enumerate}[(i)]
			\item Il existe un corps de décomposition des $f_i$ sur $K$.
			\item Soit $K \subseteq L$ un corps de décomposition des $f_i$ sur $K$ et $K \subseteq L'$ une extension dans laquelle tous les $f_i$ sont scindés.
				Alors il existe un unique morphisme de corps $\psi \colon L \to L'$ tel que $\psi_{|K} = \Id_K$.
			\item Si $f_j$ dans la famille est irréductible et $x,x'$ sont racines de $f_j$ dans $L$ et $L'$ respectivement, on peut choisir $\psi$ tel que $\psi(x) = x'$.
			\item Si $K \subseteq L'$ est aussi un corps de décomposition des $f_i$ sur $K$, tout $\psi$ comme en \textit{(ii)} est un isomorphisme.
		\end{enumerate}
	\end{pop}

	\begin{defn}
		Soit $K$ un corps.
		On appelle \textbf{clôture algébrique} de $K$, notée $K^{\alg}$, une extension $K \subseteq L$ algébrique telle que tout polynôme de $K[t]$ soit scindé sur $L$.
	\end{defn}

	\begin{rem}
		Un corps est algébriquement clos si et seulement si il est égal à sa propre clôture algébrique.
	\end{rem}

	\begin{rem}
		Une clôture algébrique de $K$ est égal au corps de décomposition de tous les polynômes de $K[t]$.
	\end{rem}

	\begin{thm}[de \textbf{Steinitz}]
		Soit $K$ un corps quelconque.
		Alors il existe une clôture algébrique de $K$ et, si $L$ et $L'$ sont deux clôtures algébrique de $K$, il existe un isomorphisme entre elles qui soit l'identité sur $K$.
		Enfin, une clôture algébrique est algébriquement close.
	\end{thm}


\subsection{Éléments et extensions algébriques séparables}

	\begin{defn}
		\textbf{Caractéristique} d'un corps $k$ : plus petit entier $p$ égal à $0$ dans $k$, ou $0$ s'il n'en existe pas.
	\end{defn}

	\begin{defn}
		Si $k$ est de caractéristique $p > 0$ on définit le \textbf{Frobenius} d'exposant $p$ : application
		$\Frob_p \colon \begin{array}{rcl} k & \to & k \\ x & \mapsto & x^p \end{array}$.
		C'est un morphisme de corps, i.e. $(x + y)^p = x^p + y^p$ et $(xy)^p = x^p y^p$.
		On note $k^p$ son image (sous-corps de $k$).
	\end{defn}

	\begin{defn}
		Soit $k$ un corps et $f \in k[t]$.
		On dit que $f$ est \textbf{séparable} s'il est premier avec sa dérivée $f'$.
		C'est équivalent à avoir des racines simples dans une extension où $f$ est scindé.
		Si $f$ est irréductible celà revient à $f' \neq 0$.
	\end{defn}
	
	\begin{pop}
		Si $k$ est de caractéristique $p = 0$, tout polynôme irréductible est séparable.
		Si $p > 0$ tout $f \in k[t]$ s'écrit de façon unique sous la forme $f(t) = f_0(t^{p^e})$ avec $e \in \N$ et $f_0' \neq 0$.
		Dans ce cas, si $f$ est séparable $e = 0$, et si $f$ est irréductible alors $f_0$ l'est aussi.
	\end{pop}
	
	\begin{lem}
		Soit $k$ de caractéristique $p > 0$ et $h \in k[t]$ tel que $\exists i \in \iniff{1}{p}, h^i \in k^p[t]$.
		Alors $h \in k^p[t]$.
	\end{lem}
	
	\begin{pop}
		Soit $k$ de caractéristique $p > 0$, $f_0 \in k[t]$ unitaire irréductible et $f(t) := f_0(t^{p^e})$ où $e > 0$.
		Alors $f$ est réductible (i.e. non irréductible) si et seulement si $f_0 \in k^p[t]$.
		Dans ce cas, on a aussi $f \in k^p[t]$.
	\end{pop}
	
	\begin{defn}
		Soit $k \subseteq K$ une extension de corps.
		Un élément $x \in K$ algébrique sur $k$ est dit \textbf{séparable} (sur $k$) si son polynôme minimal l'est.
	\end{defn}
	
	\begin{rem}
		En caractéristique $0$, tout algébrique est séparable.
		En caractéristique $p$, $\forall x$ algébrique, $\exists ! e$ tel que $x^{p^e}$ soit séparable, de degré $\deg(x)/p^e$.
		En particulier, si $p \nmid \deg(x)$, $x$ est séparable.
	\end{rem}
	
	\begin{pop}
		Soit $k \subseteq K$ une extension de caractéristique $p > 0$ et $x \in K$ algébrique sur $k$.
		Exactement l'un des deux cas suivants se produit :
		\begin{itemize}
			\item[\textbullet] $x$ est séparable, le polynôme minimal de $x^p$ sur $k$ est dans $k^p[t]$, $\deg(x^p) = \deg(x)$ et $k(x) = k(x^p)$,
			\item[\textbullet] $x$ n'est pas séparable, le polynôme minimal de $x^p$ sur $k$ n'est pas dans $k^p[t]$ et $\deg(k^p) = \deg(x)/p$.
		\end{itemize}
	\end{pop}
	
	\begin{defn}
		Une extension $k \subseteq K$ algébrique est dite \textbf{séparable} lorsque tout élément de $K$ est séparable sur $k$.
	\end{defn}
	
	\begin{pop}
		Soit $k \subseteq K$ une extension de corps finie de caractéristique $p$ telle que $K^p$ engendre $K$ comme $k$-ev.
		Alors $K$ est séparable sur $k$.
	\end{pop}
	
	\begin{pop}
		Soit $k \subseteq K$ une extension et $x_1,\ldots,x_n \in K$ tel que $\forall i \in \iniff{1}{n}$, $x_i$ est algébriques séparable sur $k(x_1,\ldots,x_{i - 1})$.
		Alors $k(x_1,\ldots,x_n)$ est séparable sur $k$.
	\end{pop}
	
	\begin{cor}
		Soit $K = k(x_i)_{i \in I}$ avec les $x_i$ algébriques séparables sur $k$.
		Alors tout $K$ est algébrique séparable sur $k$.
	\end{cor}
	
	\begin{cor}
		Soit $k \subseteq K \subseteq L$ une tour d'extensions algébriques.
		Si $K$ est séparable sur $k$ et $L$ est séparable sur $K$, alors $L$ est séparable sur $k$.
	\end{cor}
	
	\begin{rem}
		Soit $k	\subseteq K$ une extension.
		L'extension de $k$ engendrée par les éléments de $K$ algébriques séparables sur $k$ est exactement l'ensemble de ces mêmes éléments, appelé \textbf{fermeture séparable} de $k$ dans $K$.
		La fermeture séparable de $k$ dans une clôture algébrique de $k$ s'appelle \textbf{clôture séparable} de $k$.
		Si $k$ est égal à sa clôture séparable on qu'il est \textbf{séparablement clos}.
	\end{rem}
	
	\begin{rem}
		Une extension $k \subseteq K$ telle que $k$ soit égal à sa propre fermeture séparable dans $K$ est dite \textbf{purement inséparable}.
		C'est la cas si et seulement si, en notant $p > 0$ la caractéristique, le polynôme minimal sur $k$ d'un élément quelconque de $K$ est de la forme $t^{p^e} - c$ où $c \in k$.
	\end{rem}
	
	%~ \begin{defn}
		%~ \textbf{Degré séparable}
	%~ \end{defn}


\subsection{Corps parfaits, théorème de l'élément primitif}

	\begin{defn}
		Un corps $k$ de caractéristique $p$ est dit \textbf{parfait} lorsque, soit $p = 0$, soit $p > 0$ et $k^p = k$.
	\end{defn}
	
	\begin{pop}
		Un corps $k$ est parfait si et seulement si toute extension algébrique de $k$ est séparable.
	\end{pop}
	
	\begin{pop}
		Soit $k \subseteq K$ une extension algébrique avec $k$ parfait.
		Alors $K$ est aussi parfait.
	\end{pop}
	
	\begin{thm}[de l'élément primitif]
		Soit $K = k(x_1,\ldots,x_n)$ avec $x_1,\ldots,x_n$ algébriques sur $k$ et $x_2,\ldots,x_n$ séparables sur $k$.
		Alors l'extension $k \subseteq K$ est monogène, i.e. $\exists y \in K, K = k(y)$.
	\end{thm}
	
	\begin{cor}
		Toute extension finie séparable est monogène, en particulier toute extension finie d'un corps parfait.
	\end{cor}
	
	\begin{pop}
		Soit $k$ un corps parfait et $k \subseteq K$ une extension de type fini.
		Alors $\exists x_1,\ldots,x_{d + 1} \in K, K = k(x_1,\ldots,x_{d + 1})$ avec $x_1,\ldots,x_d$ algébriquement indépendants sur $k$ et $x_{d + 1}$ algébrique séparable sur $k(x_1,\ldots,x_d)$.
	\end{pop}



\subsection{Théorie de Galois}

	\begin{defn}
		Soit $K \subseteq L$ une extension algébrique.
		Deux éléments $x,x' \in L$ sont dits \textbf{conjugués} s'ils ont le même polynôme minimal sur $K$.
		C'est une relation d'équivalence qui définit des \textbf{classes de conjugaisons}.
	\end{defn}
	
	\begin{pop}
		Deux éléments $x,x' \in L$ sont conjugués lorsque tout polynôme de $K[t]$ qui s'annule sur l'un s'annule sur l'autre.
	\end{pop}
	
	\begin{rem}
		Si $x$ est séparable, son polynôme minimal sur $K$ est à racines simples dans $K^{\alg}$, donc il admet $\deg(x)$ conjugués.
	\end{rem}
	
	\begin{defn}
		Une extension $K \subseteq L$ est dite \textbf{normale} si elle vérifie une des propriétés équivalentes suivantes :
		\begin{enumerate}[(i)]
			\item tout conjugué sur $K$ dans $L^{\alg}$ d'un élément de $L$ est encore dans $L$,
			\item tout polynôme irréductible sur $K$ qui a une racine dans $L$ est scindé sur $L$,
			\item $L$ est corps de décomposition d'une famille de polynômes sur $K$,
			\item l'image de tout morphisme de corps $L \to L^{\alg}$ qui soit l'identité sur $K$ est égale à $L$.
		\end{enumerate}
	\end{defn}
	
	\begin{defn}
		Une extension algébrique est \textbf{galoisienne} si elle est à la fois normale et séparable.
	\end{defn}
	
	\begin{defn}
		Soit $K \subseteq L$ une extension galoisienne.
		On appelle \textbf{groupe de Galois} de l'extension, noté $\Gal(K \subseteq L)$, l'ensemble des automorphismes de $L$ au-dessus de $K$, i.e. les automorphismes de $L$ qui soient l'identité sur $K$.
		Lorsque $L = K^{\sep}$ (clôture séparable), on dit que $\Gal(K \subseteq L) = \Gal(K) = \Gamma_K$ est le groupe de Galois \textbf{absolu} de $K$.
	\end{defn}
	
	\begin{ex}
		$\Gal(\R \subseteq \C) = \{ \Id_\C ; x \mapsto \bar{x} \}$ et $\Gal(\F_p \subseteq \F_{p^d}) = \{ \Frob_p^i, 0 \leq i \leq d - 1 \}$.
	\end{ex}
	
	\begin{thm}
		Soit $K \subseteq L$ une extension galoisienne et $G := \Gal(K \subseteq L)$.
		Alors :
		\begin{itemize}
			\item[\textbullet] si $K \subseteq L$ est finie, alors $G$ est fini et $\card(G) = [L : K]$,
			\item[\textbullet] si $x \in L$ est fixé par tous les éléments de $G$ alors $x \in K$.
		\end{itemize}
		De plus, si on appelle $\Phi \colon E \mapsto \Gal(E \subseteq L)$ qui a un corps intermédiaire $K \subseteq E \subseteq L$ associe le groupe de Galois de l'extension $E \subseteq L$ (galoisienne) vu comme sous-groupe de $G$, on a :
		\begin{itemize}
			\item[\textbullet] $\Phi$ est une injection (décroissante pour l'inclusion), si $K \subseteq L$ est finie c'est une bijection,
			\item[\textbullet] un inverse à gauche est $H \mapsto \Fix(H) = \{ x \in L \mid \forall \sigma \in H, \sigma(x) = x \}$,
			\item[\textbullet] $\Phi(E)$ est distingué dans $G$ ssi $K \subseteq E$ est galoisienne, auquel cas $\Gal(K \subseteq E) = G/\Phi(E)$,
			\item[\textbullet] $\Phi(E_1 . E_2) = \Phi(E_1) \cap \Phi(E_2)$, et si $K \subseteq L$ est finie, $\Phi(E_1 \cap E_2)$ est le ss-gr de $G$ engendré par $\Phi(E_1)$ et $\Phi(E_2)$.
		\end{itemize}
	\end{thm}
	
	\begin{defn}
		Le \textbf{groupe de Galois d'un polynôme séparable} $f$ sur un corps $K$ est le groupe des permutations des racines de $f$ qui définissent un automorphisme du corps de décomposition.
	\end{defn}
	
	\begin{rem}
		Pour $L$ galoisienne sur $K$, les orbites $\{ \sigma(x), \sigma \in \Gal(K \subseteq L) \}$ sont exactement les classes d'équivalence pour la relation “être conjugué sur $K$”.
	\end{rem}
	
	\begin{thm}
		Soit $L$ un corps, $G$ un groupe fini d'automorphismes de $L$ et $K = \Fix_L(G) = \{ x \in L \mid \forall \sigma \in G, \sigma(x) = x \}$.
		Alors $K \subseteq L$ est une extension galoisienne de groupe de Galois $G$ et $[L : K] = \card(G)$.
	\end{thm}
	
	\begin{thm}
		Soit $G$ un groupe, $L$ un corps et $\chi_1,\ldots,\chi_n$ des caractères de $G$ dans $L$ (i.e. des morphismes $G \to L^\times$) deux à deux disctincts.
		Alors les $\chi_1,\ldots,\chi_n$ sont linéairement indépendants en tant qu'applications $G \to L$.
	\end{thm}