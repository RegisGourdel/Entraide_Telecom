\subsection{Anneaux nothérien}

	\begin{defn}
		Un idéal $I$ d'un anneau $A$ est de \textbf{type fini} s'il est engendré par un nombre fini d'éléments (équivalent à être de type fini en tant que sous-module de $A$).
	\end{defn}

	\begin{defn}
		Un anneau $A$ est dit \textbf{noethérien} lorsque tout idéal $I$ de $A$ est de type fini.
	\end{defn}

	\begin{rem}
		Un quotient d’un anneau noethérien est noethérien.
	\end{rem}

	\begin{thm}[de la base de Hilbert]
		Si $A$ est un anneau noethérien, alors l’anneau $A[t]$ des polynômes à une indéterminée sur $A$ est noethérien.
	\end{thm}

	\begin{cor}
		Soit $k$ un corps ou un anneau noethérien.
		Alors l’anneau $k[t_1,\ldots,t_n]$ des polynômes en $n$ indéterminées sur $k$ est un anneau noethérien, et plus généralement toute $k$-algèbre de type fini (comme $k$-algèbre) $k[x_1,\ldots,x_n]$ est un anneau noethérien.
	\end{cor}

\subsection{Idéaux maximaux d’anneaux de polynômes}

	\begin{lem}
		Soit $k$ un corps algébriquement clos et $K$ une extension.
		On suppose que $h_1,\ldots,h_m \in k[t_1,\ldots,t_n]$ ont un zéro commun dans $K$ (i.e $\exists z_1,\ldots,z_n \in K, \forall i, h_i(z_1,\ldots,z_n = 0$).
		Alors ils en ont un dans $k$.
	\end{lem}

	\begin{note}
		Soit $k$ un corps et $(x_1,\ldots,x_n) \in k^n$. On note
		$$\mathfrak{m}_{(x_1,\ldots,x_n)} := \{ f \in k[t_1,\ldots,t_n] \mid f(x_1,\ldots,x_n) = 0 \} = (t_1 - x_1,\ldots,t_n - x_n)\ .$$
	\end{note}

	\begin{pop}
		Soit $k$ un corps algébriquement clos.
		Les idéaux maximaux de $k[t_1,\ldots,t_n]$ sont exactement les idéaux $\mathfrak{m}_{(x_1,\ldots,x_n)}$.
	\end{pop}
	
	\begin{pop}[lemme de Zariski]
		Soit $k$ un corps et $K$ une extension de type fini comme $k$-algèbre.
		Alors $k$ est en fait une extension finie.
	\end{pop}


\subsection{Le Nullstellensatz}

	\begin{pop}[Nullstellensatz faible]
		Soient $h_1,\ldots,h_m \in k[t_1,\ldots,t_n]$ avec $k$ algébriquement clos.
		Si $h_1,\ldots,h_m$ n'engendrent pas l'idéal unité, alors ils ont un zéro commun dans $k$: $\exists x_1,\ldots,x_n \in k, \forall i, h_i(x_1,\ldots,x_n) = 0$.
	\end{pop}

	\begin{pop}[Nullstellensatz fort]
		Soient $g, h_1,\ldots,h_m \in k[t_1,\ldots,t_n]$ avec $k$ algébriquement clos.
		Si $g$ s'annule sur tous les zéros commun de $h_1,\ldots,h_m$ alors $\exists l \in \N, g^l \in (h_1,\ldots,h_m)$ (idéal engendré).
	\end{pop}


\subsection{Fermés de Zariski}

	\begin{defn}
		Un idéal $\mathfrak{r}$ d'un anneau $A$ est dit \textbf{radical} lorsque $A / \mathfrak{r}$ est réduit, i.e $\forall x \in A, \forall n \in \N, x^n \in \mathfrak{r} \implies x \in \mathfrak{r}$.
	\end{defn}
	
	\begin{defn}
		Soit $I$ un idéal de $A$.
		Le \textbf{radical de $I$} est $\surd I = \{ x \in A \mid \exists n \in \N, x^n \in I \}$.
		C'est un idéal radical.
	\end{defn}

	Un idéal premier, et a fortiori un idéal maximal, est en particulier un idéal radical.

	Dans ce qui suit on note $k$ un corps et $k^{\alg}$ une clôture algébrique.

	\begin{note}
		Soit $\mathscr{F} \subset k[t_1,\ldots,t_n]$.
		On pose $Z(\mathscr{F}) := \{ (x_1,\ldots,x_d) \in (k^{\alg})^d \mid \forall f \in \mathscr{F}, f(x_1,\ldots,x_d) = 0 \}$.
	\end{note}
	
	\begin{defn}
		On appelle \textbf{fermé de Zariski} tout ensemble de la forme $Z(\mathscr{F})$.
	\end{defn}
	
	\begin{rem}
		$Z$ est décroissante pour l'inclusion et on peut toujours supposer que $\mathscr{F}$ est un idéal radical.
	\end{rem}

	\begin{defn}
		Un fermé de Zariski de la forme $Z(f) = Z(\{ f \})$ est appelé une \textbf{hypersurface}.
	\end{defn}

	\begin{rem}
		Le vide, $(k^{\alg})^d$ et les singletons sont des fermés de Zariski.
	\end{rem}

	\begin{note}
		Soit $E \subset (k^{\alg})^d$.
		On pose $\mathfrak{J}(E) := \{ f \in k[t_1,\ldots,t_n] \mid \forall (x_1,\ldots,x_d) \in E, f(x_1,\ldots,x_d) = 0 \}$.
	\end{note}

	\begin{rem}
		$\mathfrak{J}(E)$ est un idéal radical, $\mathfrak{J}$ est décroissante pour l'inclusion, $\mathfrak{J}(E) = \bigcap_{x \in E} \mathfrak{M}_x$ où $\mathfrak{M}_x = \mathfrak{J}(\{ x \})$ et en particulier $\mathfrak{J}(E) \neq k[t_1,\ldots,t_d] \iff E = \emptyset$.
		De plus $\mathfrak{J} \left( (k^{\alg})^d \right) = \{ 0 \}$.
	\end{rem}

	\begin{pop}
		Soit $E \subset (k^{\alg})^d$ et $\mathscr{F} \subset k[t_1,\ldots,t_d]$.
		On a $E \subset Z(\mathscr{F}) \iff \mathscr{F} \subset \mathfrak{J}(E)$.
	\end{pop}
	
	\begin{pop}
		Une partie $E \subset (k^{\alg})^d$ vérifie $E = Z(\mathfrak{J}(E))$ si et seulement si c'est un fermé de Zariski.
	\end{pop}
	
	\begin{pop}
		Soit $I$ un idéal de $k[t_1,\ldots,t_d]$ et $E \subset (k^{\alg})^d$.
		Alors $\mathfrak{J}(Z(I)) = \surd I$ et $Z(\mathfrak{J}(E))$ est le plus petit fermé de Zariski défini sur $k$ qui contient $E$.
		De plus, $Z$ et $\mathfrak{J}$ définissent des bijections réciproques décroissantes entre idéaux radicaux de $k[t_1,\ldots,t_d]$ et fermés de Zariski de $(k^{\alg})^d$ définis sur $k$.
	\end{pop}
	
	\begin{defn}
		Les éléments de $Z(I) \cap k^d$ sont appelés \textbf{points rationnels} de $Z(I)$.
		Ceux dans $Z(I) \cap ((k^{\alg})^d \setminus k^d)$ sont appelés \textbf{points géométriques}.
		Enfin on appelle \textbf{point fermé} les $Z(\mathfrak{m})$ avec $\mathfrak{m}$ un idéal maximal de $k[t_1,\ldots,t_d]$ et $I \subset \mathfrak{m}$.
	\end{defn}
	
	\begin{defn}
		Le corps $\varkappa_{\mathfrak{m}} = k[t_1,\ldots,t_d] / \mathfrak{m}$ s'appelle \textbf{corps résiduel} de $Z(\mathfrak{m})$.
		La classe modulo $\mathfrak{m}$ d'un polynôme s'appelle \textbf{évaluation} du polynôme en $Z(\mathfrak{m})$ et $[\varkappa_{\mathfrak{m}} : k]$ est appelé \textbf{degré} de $Z(\mathfrak{m})$.
	\end{defn}
	
	\begin{defn}
		Soit $I$ un idéal radical de $k[t_1,\ldots,t_d]$.
		On appelle $k[t_1,\ldots,t_d] / I$ l'\textbf{anneau des fonctions régulières} de $Z(I)$.
		Une fonction régulière s'identifie à la restriction à $Z(I)$ d'un polynôme.
	\end{defn}
	
	\begin{defn}
		Un fermé de Zariski $Z(I)$ est \textbf{irréductible} si $\forall I_1,I_2, \left( Z(I) = Z(I_1) \cup Z(I_2) \right) \implies (Z(I_1) = Z(I)\ \text{ou}\ Z(I_2) = Z(I))$.
	\end{defn}
	
	\begin{pop}
		Soit $I$ un idéal radical.
		Alors $Z(I)$ est irréductible si et seulement si $I$ est premier.
		En particulier une hypersurface $Z(f)$ est irréductible si et seulement si $f$ est nul ou irréductible.
	\end{pop}
	
	\begin{defn}
		Si $f \in k[t_1,\ldots,t_d]$ est irréductible dans $k^{\alg}[t_1,\ldots,t_d]$ il est dit \textbf{géométriquement irréductible} ou \textbf{absolument irréductible}.
		Les seuls polynômes en une variable géométriquement irréductibles sont ceux de degré 1.
	\end{defn}
	
	\begin{defn}
		Soit $I$ idéal radical de $k[t_1,\ldots,t_d]$, $Z(I)$ est dit géométriquement (ou absolument) irréductible si l'idéal $I.k^{\alg}$ engendré par $I$ dans $k^{\alg}[t_1,\ldots,t_d]$ est premier.
		Si $I = (f)$ est principal, c'est équivalent à avoir $f$ nul ou géométriquement irréductible.
	\end{defn}


\subsection{Extension des scalaires des algèbres sur un corps}

	\begin{defn}
		Soit $k \subseteq k'$ une extension de corps, $V$ un $k$-ev et $(e_i)$ une base de $V$.
		Le $k'$-ev de base $(e_i)$ est appelé \textbf{extension des scalaires} de $V$ de $k$ à $k'$, noté $V \otimes_k k'$ et $\dim(V \otimes_k k') = \dim(V)$.
	\end{defn}
	
	\begin{note}
		Soit $\imath$ l'application qui à $\sum \lambda_i e_i \in V$ associe $\sum \lambda_i e_i$ où les $\lambda_i$ sont vus dans $k'$, et dont l'image engendre $V'$ comme $k'$-ev.
		On note $\forall x \in V, \forall c \in k', x \otimes c := c \cdot \imath(x)$.
	\end{note}
	
	On peut considérer $V \otimes_k k'$ comme un $k$-ev dont une base est donnée par les $(e_i \otimes b_j)_{(i,j)}$, avec $(b_j)_j$ une base de $k'$ comme $k$-ev.
	
	\lightning\ Les éléments de $V \otimes _k k'$ ne sont pas tous de la forme $x \otimes c$ mais ceux ci engendrent $V \otimes _k k'$.
	
	\begin{defn}
		Le \textbf{produit tensoriel} $V \otimes_k W$ de deux ev $V$ et $W$ sur un corps $k$ est l'espace vectoriel dont une base est le produit d'une base de $V$ et d'une base de $W$.
	\end{defn}
	
	\begin{pop}
		Soit $k \subseteq k'$ une extension de corps, $V$ un $k$-ev, $U$ un sous-$k$-ev de $V$ et $W := V/U$.
		Notons $U',V',W'$ les extensions des scalaires de $U,V,W$ de $k$ à $k'$, et $U' \to V', V' \to W'$ les applications $k$-linéaires obtenues par extension des scalaires à partir de l'injection d'incluseion (i.e. l'identité) $U \to V$ et la surjection canonique $V \to W$.
		Alors $V' \to W'$ est surjective, $\Ker(V' \to W') = \im(U' \to V')$ et $U' \to V'$ est injective.
	\end{pop}
	
	\begin{pop}
		Soit $k$ un corps, $A$ une $k$-algèbre et $k'$ une extension algébrique de $k$.
		Supposons $A \otimes_k k'$ intègre, alors $\Frac(A \otimes_k k') \simeq \Frac(A) \otimes_k k'$.
		De plus $\Frac(A)$ et $k'$ sont linéairement disjointes comme extensions de $k$ contenues dans $\Frac(A \otimes_k k')$ et $\Frac(A \otimes_k k') = \Frac(A).k'$.
	\end{pop}