L'identité est une question majeure depuis quelques décennies, à la fois en philosophie et en sociologie.
On ne propose pas ici de définition unifié de l'identité.

\subsection{Première condition de l'identité : l'unité psychologique}
	Selon l'approche de la philosophie classique l'identité repose sur une unité intérieure, psychologique.
	Elle renvoie alors au « moi », « je », « soi ».
	
	Aussi variés que soient mes facultés mentales et mes états psychologiques, ils semblent tous renvoyer au même pôle de subjectivité que constitue ce moi.	
	
	Pour montrer l'importance du “moi” on peut exhiber une menace contre l'unité psychologique : la schizophrénie, fractionnement de l'unité psychologique.
	Dans ce cas, un des symptômes les plus remarquable est que le patient a l'impression d'entendre des voix.

	On trouve un grand nombre de variations parmi ce symptôme.
	Cette voix peut-être multiple, celle d'une personne connue, d'une personne morte, d'un personnage fictif.
	Elles peuvent s'adresser à lui directement où en parler à la troisième personne.
	Le propos de ces voix consiste souvent en des reproches envers le sujet.
	Le sujet peut avoir l'impression que ces voix sont intérieurs ou bien extérieures.
	Les réactions du sujet varient également.

	La connaissance de ce phénomène a pu progresser grâce à l'imagerie cérébrale.
	On constate alors, dans le cas d'un sujet en train de vivre un épisode d'hallucination verbale, que certaines aires liées à la perception du langage sont actives.
	De plus, certaines aires de production du langage sont actives.
	Cependant une certaine zone du cerveau, habituellement liée à la reconnaissance de son propre discours, n'est pas activée.
	Le sujet ne s'attribue donc pas l'origine du discours.
	Ceci est confirmé par le fait que les schizophrènes ont généralement du mal à s'attribuer la responsabilité non seulement de leur parole mais aussi de leurs actes.

	Par ailleurs le problème des schizophrènes n'est pas tant qu'ils ont plusieurs personnalités mais plutôt qu'ils ne parviennent pas à en fixer une.

	Ceci nous permet donc d'entrevoir que l'unité psychologique qui peut sembler évidente n'est pas forcément acquise et repose sur différents processus biologiques.
	Avoir le sentiment d'être un “je” requiert donc la présence de site cérébraux dispersés à travers le cerveau et qui ne possèdent probablement pas de zones jouant le rôle de superviseur, même si le sentiment résultant est unifié.
	C'est le résultat de la coopération de ces zones qui produit le sentiment de notre identité.

\subsection{Deuxième condition de l'identité : l'unité anatomique}
	On peut noter premièrement que l'homme n'a pas les propriétés de reconstitution que possèdent des organismes comme les plantes ou les vers de terre, ce qui montre que l'unité anatomique de l'homme se définit d'une manière différente.

	On doit à Jacques Lakan un article sur la notion d'identité basé sur l'expérience du miroir.
	Selon lui les sensations que l'on peut avoir auparavant ne produisent pour nous qu'une image morcelée.
	Aux alentours de 18 mois, le bébé placé face à un miroir est alors capable de se reconnaître.

	C'est en fait le cas avec peu d'autres espèces, ce que l'on étudie à partir du test de Gallup.

	L'enfant va notamment reconnaître son visage.
	Selon Lakan le miroir permet aussi d'offrir l'image d'un corps entièrement délimité par un contour.
	La représentation est donc donnée par image unifiée une extériorité.
	Selon Lakan cette expérience fondatrice de l'identité serait également la base du sentiment narcissique.

	On pourra tout de même critiquer cette approche en remarquant que différentes civilisations ne possédaient ni miroirs, ni surfaces d'eau suffisamment claires pour que l'enfant puisse faire l'expérience de son identité physique.\\

	On pourra exhiber là encore la encore une pathologie révélatrice : la dysmorphophobie.
	Le symptôme de cette maladie, qui peut être bénigne ou bien grave, est que le sujet n'aime pas une partie de son corps, voire croit que celle-ci ne lui appartient pas.
	Dans les cas bénins, un sujet peut juste être amené à cacher ses oreilles.
	Dans les cas extrêmes, le sujet peut alors demander une amputation, voire en tenter une par lui-même.

	Cela permet donc encore une fois de constater que le sentiment d'intégrité corporelle n'est pas tout à fait évident.
	Des expériences ou traumatismes particuliers durant l'enfance peuvent ainsi résulter en une construction du sentiment d'intégrité physique en excluant certaines parties de son corps.
	Le sujet devient alors convaincu qu'un ou plusieurs de « ses » membres ne lui appartiennent pas.

	Un autre cas où le sentiment d'identité physique ne s'opère pas est celui des transsexuels.
	Il est dans leur cas plus facile d'accéder à une opération qui leur permette de modifier leur identité corporelle.

\subsection{Troisième condition de l'identité : la permanence dans le temps}
	Nous avons l'habitude d'utiliser le “je” pour nous désigner à travers le temps avec une certaine continuité.
	On peut comprendre en cela que les changements qui s'opèrent sur nous sont assez lents.
	On peut aussi le comprendre comme le fait que, malgré les changements qui s'opèrent, il existe en nous un facteur invariant qui nous définit.

	Pour des modifications mineures (physiques notamment) de l'individu, on considère que son identité ne change pas.
	Toutefois dans le cas de changements importants, de personnalité par exemple, on peut commencer à hésiter à affirmer qu'une personne est la même que précédemment.

	Exemple du bateau de Thésée : ce bateau est constitué de 1095 lattes, soit $3 \times 365$, et les marins remplacent chaque jour une latte.
	Après trois ans on ne retrouve alors plus sur le bateau aucune latte du bateau initial.
	On peut alors se demander si le bateau obtenu après ces trois ans est encore le même que celui du départ.

	Le lien avec l'être humain peut se faire en ce que nos cellules se renouvellent régulièrement, plus ou moins vite selon les parties du corps, de sorte que, en l'espace de trois mois, près de 90\% de notre corps a changé.

	%(D. Hume ?)

	Le lien peut se faire également au niveau du connectome, notion sur laquelle spéculent les transhumanistes par exemple, qui décrit l'ensemble des connexions synaptiques de notre cerveau.
	Il est supposé alors qu'une cartographie complète de notre cerveau pourrait permettre de nous reconstituer dans le futur.
	Toutefois, pour la plupart des personnes, ce ne serait pas exactement nous.

	À l'inverse, pour la cryogénie, on considère que c'est bien nous qui serions ressuscités.\\

	Le philosophe John Locke soutenait que l'identité de l'individu s'appuie essentiellement sur la mémoire, capacité fondamentale qui relie les différents moments de mon expérience.
	On pourrait alors presque dire “je suis une mémoire”.

	On peut toutefois craindre que l'approche de Locke ne soit assez glissante en ce qu'elle néglige tout ce qui relève de l'inconscient, des expériences vécues sans pour autant être inscrites dans la mémoire.

\subsection{Quatrième condition de l'identité : l'unicité}
	L'identité se conçoit difficilement sans la possibilité d'être identifié de façon unique.

	Il existe différents marquages physiques qui nous permettent de nous identifier de façon quasiment unique : empreintes digitales, génome...\\

	On trouve aussi des identifiants administratifs traditionnels, la carte d'identité notamment.

	On notera que cet identifiant peut être sujet à des variations : au Laos on peut changer de nom si l'on survit à une maladie grave et chez les Inuit on possède un nom pour la saison de nuit et un autre pour la saison de jour.
	De même, dans l'ensemble des sociétés patriarcales, la femme change ou peut changer de nom au moment de son mariage, signifiant ainsi que cet événement possède une importance telle qu'elle affecte son identité.\\

	L'affirmation de l'identité de la personne est aujourd'hui mise en avant dans la publicité à travers les produits de consommation.
	Le philosophe et sociologue français Jean Baudrillard écrit que ceci a donné lieu à un marché de la personnalisation.
	Le système marchand s'est alors transformé en un gigantesque pourvoyeur de pseudo-identité.
	On subvient de façon illusoire à un besoin de personnalisation qui est entretenu par la publicité.

	L'histoire de l'économie de marche a confirmé ce constat fait par Jean Baudrillard, qui date des années 1960.
	Historiquement, l'idée d'identité est associé à une émancipation, mais aujourd'hui cette façon de cultiver un individualisme de masse est devenu un gigantesque gisement marchand.\\

\subsection{L'identité sociale}
	Dans l'histoire de la philosophie, et au XX\up{e} siècle, c'est Sartre qui a conféré un rôle privilégié à la présence d'autrui dans la constitution de l'identité.
	Exemple : garçon de café observé par Sartre qui semble surjouer son rôle.
	Ceci constitue pour lui une aliénation : une façon de se laisser enfermer dans un rôle qui nous est étranger.
	Le reste de la société nous voit alors à travers ce rôle.
	Nous avons tendance à endosser le(s) rôle(s) que les autres nous attribuent, au point de nous confondre avec les masques sociaux que nous revêtons.

	Ceci constitue pour Sartre la menace par excellence sur l'identité.\\

	Problème : depuis les années 70, la psychologie sociale a contribué à montrer que l’individu construit également son identité sociale à travers les groupes de référence auxquels il puise son estime de lui-même.
	Henri Tajfel, en particulier, a indiqué comment ces composantes collectives de l'identité pouvaient construire des « identités de rattrapage ».

	On peut remarquer notamment que, pour se décrire, les gens tendent à mettre en avant leur appartenance à différents groupes.
	Pour Tajfel, l'individu aura tendance en général a choisir le groupe de plus valorisant, c'est-à-dire celui qui lui donnera l'estime de soi la plus haute.
	L'individu aura alors tendance à utiliser le “nous” pour compenser un manque social personnel.

	L'un des danger du recours à ces composantes collectives de l'identité est qu'il peut être le terreau d'identification psychiques identitaires.

\subsection{L'identité professionnelle menacée par l'aliénation}
	La paternité du terme d'aliénation, évoqué plus haut avec Sartre, revient plutôt à Karl Marx.
	Pour lui le travailleur subit une aliénation car tout ce qu'il contribue à faire est contraire à ce qu'il est.
	L'ouvrier se sent étranger à lui-même car il ne peut nullement s'identifier à son activité.

	%(E. Mayo)

	Cette notion d'aliénation a été renouvelée dans les années 1980 par Arlie Russel Hochschild, concernant en particulier les métiers de services avec un important contact avec le client.
	Hochschild relève que ces métiers requièrent une part importante de présentation de soi : sourire...
	Cet aspect est nommé aliénation émotionnelle et irait au-delà d'une simple hypocrisie sociale.
	Pour elle, ces mimiques et expressions deviennent un prolongement des moyens pour contrôler l'humeur des clients.

	Ce comportement qui consiste en société à se placer dans une humeur adaptée à une situation (mariage,...) est ce que certains métiers exigent en permanence.
	Hochschild s'est ainsi intéressée à la formation des hôtesses de l'air, celles-ci devant avoir l'air souriante, chaleureuse, tout en donnant l'impression que ces sourires ne sont pas hypocrites.
	Ils ne s'agit pas alors de feindre ses sentiments mais de manager ses émotions pour rester dans de bonnes dispositions même avec des passagers antipathiques.
	Ces hôtesses de l'air sont donc soumises à des contraintes émotionnelles à long terme, pouvant occasionner des burn-out émotionnels qui sont dénoncés par Hochschild.
	Le risque selon Hochschild est que l'hôtesse en perde alors ses émotions authentiques, de même que, chez Marx, le travail a un effet sur le travailleur : ne plus être réellement capable de différencier le vrai du faux parmi leurs émotions.
