\subsection{Les pathologies psychiques}

	On connaît aujourd'hui un certain nombres de pathologies mentales qui sont spécifiques à une zone géographique donnée, et donc imputées à des causes culturelles.
	On trouve ainsi l'\textit{amok} chez des populations en Indonésie et Malaisie (référencé dans le DSM) ou le \textit{koro}.

	Ces maladies peuvent aussi être localisées dans le temps.
	On peut ainsi évoquer l'hystérie qui, selon Freud, fait partie des maladies qui ont permis de lancer la psychanalyse.
	Il a alors été remarqué que, après un temps, cette pathologie s'est faite beaucoup plus rare.

	Il a été remarqué par Alain Ehrenberg que, en son temps, beaucoup de pathologies psychiques passées étaient liées à la culpabilité (peur de sortir d'un cadre moral) alors que les plus récentes étaient liées à l'infériorité.
	Les variations de prévalence des pathologies psychiques posent des problèmes d'interprétation assez délicats.
	Par exemple, durant les dernières décennies du XX\up{e} siècle, de nombreux psychologues ont noté une diffusion de troubles associés à des formes d'anxiété sociales : sentiments d'inadéquation, d'échec, d'infériorité...

	On explorera alors plusieurs pistes dans des domaines divers pour expliquer ce déplacement.

\subsection{Philosophie politique : Tocqueville}

	Tocqueville est l'auteur de \textit{De la démocratie en Amérique}.
	Il a su saisir l'impact des valeurs égalitaires sur les mœurs, la signification de l'égalité étant loin d'être seulement politique.
	Il s'agit en effet également d'une égalité des conditions.
	Tocqueville montre à quel point ce système entre en opposition avec le régime hiérarchique des siècles passés, comme celui ayant cours en France avec la noblesse.

	Selon Tocqueville les barrières entre les rangs sont visibles (vestimentaires,...) ce qui rend les risques de confusions minimes.
	Il y a une faible communication entre les différentes classes sociales.
	À l'inverse, l'égalité des conditions, malgré l'absence d'égalité réelle, permet aux inividus de se sentir semblables les uns aux autres et leur permet d'envisager leur rapport d'une façon égalitaire.
	Parce qu'il n'y a plus de frontière hermétique, le désir d'ascension est alors autorisé, l'ambition est plus répandue.
	Auparavant ce sentiment envieux était borné par les frontières quasi infranchissables que constituait les différences entre castes sociales, qui existent en fait en parallèle les unes des autres.
	Cette ambition plus répandue libère alors les jalousies sociales.

	Chacun va désormais tourner son regard vers ceux qui sont au-dessus mais désormais accessibles, tout en s'inquiétant de la préservation de ses acquis.
	L'égalité sociale a donc un coût majeur qui est une plus grande sensibilité aux inégalités et différences qui subsistent.

	Cette nouvelle horizontalité est telle que l'on envie maintenant ceux qui sont proches de nous et nous dépassent de peu.
	Or beaucoup de monde nous dépasse de peu désormais.

	La démocratie rapproche donc les inégaux, en leur donnant conscience de ces inégalités.
	
	\begin{aquote}{Alexis de \bsc{Tocqueville}, \textit{De la démocratie en Amérique}, II, II, 13}
	“Ils ont détruit les privilèges gênants de quelques uns de leurs semblables ; ils rencontrent la concurrence de tous [...] Quand l'inégalité est la loi commune d'une société, les plus fortes inégalités ne frappent point l'œil ; quand tout est à peu près de même niveau, les moindres le blessent.”
	\end{aquote}

	À noter que Tocqueville n'est pas contre la démocratie et reconnaît par ailleurs le progrès politique qu'elle constitue.

\subsection{Sociologie : Veblen}

	Thorstein B. Veblen est un sociologue américain d'origine norvégienne.

	Il décrit un processus dans lequel les classes supérieures sont sans cesse copiées et imitées, dans une forme de descente en cascade des pratiques et comportements.
	Elles ont donc besoin de se renouveler suffisamment vite pour continuer à distinguer.

	Nombres d'études sociologiques montrent que ce soucis de distinction et le désir d'imitation entretiennent un processus social apparemment sans fin, manifeste dans certains phénomènes de mode : chaque classe sociale se préoccupe de se distinguer de celle d'en-dessous et de rattraper celle d'au-dessus.

	En France on peut trouver cela à l'intérieur de la noblesse durant la monarchie pré-révolutionnaire, étendu ensuite aux bourgeois.
	En Amérique Veblen note que ces mécanismes se répercutent dans la bourgeoisie américaine.
	Ce processus est lié aussi à l'économie de marché : le développement des pratiques de consommation est tel que ce sont de plus en plus les biens de consommations qui vont permettre de “tenir son rang”.
	L'attitude et les manières sont alors délaissées dans cette pratique de course entre classe sociales.

	C'est donc ce mécanisme qui va constituer une force d'entraînement majeur de l'économie de marché, permettant de maintenir un processus de croissance.

	Veblen parle de \textit{invidious comparison} (comparaison envieuse).
	Tant que cette comparaison est défavorable à un individu, il vit dans l'insatisfaction.
	On croit et intériorise alors que le bonheur consiste en le fait d'avoir quelques chose que d'autres ne peuvent pas posséder.
	Les individus fondent leur valeur relativement aux autres, chacun se jugeant en fonction de ses pratiques de consommation.
	
	Ces mécanismes sociaux sont donc une aubaine pour l'économie de marché mais ils fragilisent l'estime de soi des individus, appellés à être jugés en fonction de leurs acquisition relatives.
	
	\begin{aquote}{Thorstein B. \bsc{Veblen}, \textit{Theory of the leisure class}, 1899}
	“Il est devenu indispensable d’acquérir des biens pour conserver sa réputation. Les membres de la communauté qui n’y parviennent pas baisseront dans l’estime de leurs semblables; et par conséquent ils baisseront aussi dans leur propre estime.”
	\end{aquote}

\subsection{Philosophie morale : Dupuy}

	Jean-Pierre Dupuy est un philosophe français (encore vivant) qui s'est intéressé à des petites communautés rurales du Mexique dans lesquelles il n'y a pas de compétition vestimentaire ou démonstrative de manière générale.
	Au contraire on y trouve un conformisme qui les conduit à éviter toute ostentation et préserver une morne modestie.
	Par peur de l'\textit{invidia}, chacun cherchera à minimiser les avantages qu'il a sur les autres.

	Ceux qui se retrouvent en position de supériorité vont même chercher à attribuer ce décalage à des facteurs extérieurs : chance, faveur des dieux...
	Les sentiments envieux ne sont pas absent mais jugulés par cette culture.

	Ceux qui se trouvent en état d'infériorité peuvent attribuer leur infortune à une cause située en dehors de leur sphère de contrôle : les dieux, l'ordre cosmique, le destin...
	L'idée de Dupuy est alors que, dans nos sociétés où l'autonomie individuelle est une valeur cardinale, c'est le contraire.
	Il n'y a en effet plus d'ordre transcendant sur lequel se reposer.
	Chacun se retrouve alors obligé d'assumer le responsabilité de ce qu'il est.

	Dans une société comme la notre, selon Dupuy, la gloire du vainqueur et la honte du perdant leur sont également imputables.
	L'individu sera amené à incriminer sa paresse, son incompétence...

	Dupuy est donc en désaccord avec la Tocqueville : c'est la liberté et non l'égalité qui serait anxiogène.

\subsection{Psychologie du développement}

	Plusieurs théoriciens marxistes se sont intéressés aux “perversions du système d'échange capitaliste”.
	En effet l'argent est devenu un système de mesure entre les individus eux-mêmes, en plus de servir de mesure dans l'économie.

	Si cela est possible c'est que, de façon générale, les individus acceptent cette classification chiffrée et symbolique.
	L'intérêt des individus est dévié vers une représentation abstraite de la supériorité.

	Des recherches suggèrent que c'est à l'école qu'est contracté ce besoin étrange de recourir à des mesures pour se comparer.
	C'est en effet là que les individus rencontrent pour la première fois, à travers les notes par exemple, un ordre basé sur un tel classement.
	Des dispositifs d’évaluation leur sont appliqués en prétendant objectiver leur valeur personnelle. 

	On peut imaginer que ce système de valorisation scolaire n'est pas sans conséquence sur le développement psychologique de l'enfant.
	Des travaux des années 1980 montrent que l'estime de soi de l'enfant baisse au cours des premières années de sa scolarisation.

	Avant l'entrée à l'école, l'estime de l'enfant serait non comparative, et basé uniquement sur ce qu'il arrive à faire ou pas.
	À l'école cela diffère en ce que les résultats sont comparés à ceux des autres (J. Chafel).

	Les performances sont alors elles-mêmes rabattues sur les notes.
	C’est donc peut-être dans le système éducatif que les individus apprennent à suspendre leur estime d’eux-mêmes à des systèmes de mesure... qui les accompagneront ensuite tout au long de leur vie.

\subsection{Psychanalyse : Adler}

	Alfred Adler détecte une catégorie de névrosés qui semblent construire toute leur existence comme un système de compensation contre une déficience vécue.
	Il forge alors le concept de “complexe d'infériorité”.

	Il étend ensuite sa découverte en faisant l'hypothèse que les sentiments d'infériorité sont l'une des sources de tourment les plus universelles.
	
	\begin{aquote}{Alfred \bsc{Adler}, \textit{Le tempérament nerveux}, 1911}
	“Toujours et dans tous les cas son vouloir et sa pensée reposent sur une base formée par un sentiment d'infériorité, sentiment relatif, produit d'une comparaison que le patient établit entre lui et d'autres personnes.”
	\end{aquote}

\subsection{Théorie des médias}

	Avec l'accès aux médias de masse, les individus sont exposés à une multiplication et un “réhaussement” de leurs cibles comparatives usuelles.

	Les spectateurs sont ainsi plus régulièrement conduits à se mesurer (consciemment ou pas) à des images modèles et à se découvrir en défaut.
	Exemple : image du corps féminin.

\subsection{Épilogue}

	Pourquoi la dimension de la hauteur est-elle privilégiée dans l'expression de cette problématique ?

	\begin{itemize}
	\item Anatomie : positions des parties du corps, importance de la position debout.
	\item Configurations humaines élémentaires : combat corps-à-corps, position de l'enfant par rapport à ses parents.
	\item Relation à l'environnement naturel : ciel, montagnes...
	\end{itemize}
