Le droit social inclut le droit du travail et le droit de la sécurité sociale.
Le droit du travail régit la relation entre employeur privé et salarié.
Il ne concerne pas le domaine public (géré par le droit de la fonction publique).

\paragraph{Interférences entre droits :}
	\begin{itemize}
		\item[\textbullet] Salarié hybride dit « protégé », du style syndicat, CE, délégué du personnel : dépend du droit de la fonction publique.
		\item[\textbullet] Lors d’accident mortel sur un chantier : droit pénal (le patron est responsable pénalement).
	\end{itemize}

\paragraph{Partenaires sociaux :} médiateur pour négociation salarié/employeur, agit pour améliorer les conditions de travail.
	Les deux principaux types de partenaires sont :
	\begin{itemize}
		\item[\textbullet] les organisations patronales (syndicats) : MEDEF, UPA, CGPME,
		\item[\textbullet] les organisations syndicales représentatives pour les employés : CGT, CFDT, FO, CGCCFE.
	\end{itemize}
