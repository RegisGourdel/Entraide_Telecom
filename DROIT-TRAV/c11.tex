\paragraph{Problème :}
Une entreprise d’informatique veut délocaliser en Pologne.
Une partie des employés est en CDI, le reste en CDD.

Pour les employés : comment agir en groupe pour éviter la délocalisation ?
Pour l'entreprise : comment mener à bien son plan ?

On peut délocaliser même si on n’est pas en perte sur l’usine. Pour rester compétitif dans une économie mondialisée, il est possible que l’on ait à délocaliser une usine.
Si le motif n’est pas jugé valable, on peut mener des mesures légales.

Différence entre déménagement et délocalisation : le déménagement reste en France alors que avec la délocalisation on change de pays.

Pour un licenciement de plus de 10 salariés un plan de sauvegarde de l'emploi (PSE, ex plan social) est obligatoire : recherche d'un replacement avant licenciement économique.
L’entreprise peut consulter en faisant un vote mais c'est juste consultatif il n'y a pas d'obligation de suivre le vote.

Plan de départ volontaire est possible : sauf cas particuliers les employés ne seront pas déplacés en Pologne, donc plutôt que des licenciements économiques on préfère faire des départs volontaires avec des indemnités et/ou des avantages supérieurs à ce que la loi oblige.
Mais il peut y avoir trop peu de volontaire, ou au contraire trop.

\paragraph{Actions de l'entreprise :}
	\begin{itemize}
	\item[\textbullet] Reclassement des salariés de manière interne ou externe.
	\item[\textbullet] Aide à réinsertion.
	\item[\textbullet] Lettres de recommandation.
	\item[\textbullet] 2/3 de salaire versé pendant 1 an pour trouver du travail.
	\item[\textbullet] Mise à disposition de psychologues pour pouvoir accompagner les salariés.
	\item[\textbullet] On peut faire une délocalisation meme si on fait des bénéfices car dans le cadre de la mondialisation il faut rester compétitif.
	\end{itemize}


\paragraph{Réaction des salariés :}
	\begin{itemize}
	\item[\textbullet] Demande d'échéance, licenciements ou déplacement ? Quelles indemnités ?
	\item[\textbullet] Quid des personnes soumises à des clauses d’exclusivité ou de non mobilité ?
	\item[\textbullet] Ne pas partir du fait que c'est acquis ! Demander pourquoi etc... Il faut essayer de gagner du temps pour pouvoir mener une négociation.
	\item[\textbullet] Si motif économique, regarder les bénéfices etc... pour voir si c'est justifié.
	\item[\textbullet] Reprise possible ?
	\item[\textbullet] Il faut d'abord chercher les causes avant de parler des mesures.
	\item[\textbullet] Il faut aussi comparer au niveau de la filiale et non du groune entier : une partie du groupe peut être en déficit même si tout le groupe ne l'est pas. 
	\item[\textbullet] Si le motif de la délocalisation n'est pas legitime/légale on peut mener des mesures légales.
	\item[\textbullet] Appel aux syndicats.
	\item[\textbullet] Appel à l'aide auprès de l'État.
	\item[\textbullet] Retardement de la délocalisation.
	\end{itemize}
	
	D'après la loi travail on peut maintenant soumettre au référendum une méthode d'action (par exemple changer les salaires).
	En France on n'est pas dans de la co-gestion donc les employés ne peuvent pas proposer un plan de remaniement.

	
\paragraph{}
Le plan de sauvegarde de l'emploi est transmis à l'instance administrative et l'inspection du travail va tout vérifier : motif, effort suffisant mis en œuvre pour sauvegarder les emplois, moyens financiers engagés.
Un échange a lieu sur les différents points du plan.
Une fois que l'inspection du travail à homologué le plan, l'employeur peut le mettre en oeuvre.
Cependant on peut encore le contester \textrightarrow\ faire appel au tribunal administratif.

Si les salariés ne veulent pas l'accepter ils ont des moyens de se battre.
On peut faire grève : c'est un droit constitutionnel, mais on n'est pas payé.
Par contre il faut qu'elle soit collective pour qu'elle ait une légitimité et pour être protégé.
La grève perlée (exercer ses fonctions tout en visant à diminuer sa productivité ou son efficacité) est interdite.

Souvent il faut être sur le terrain pour mobiliser / signer des pétitions...
Le but de la grève c'est qu'elle se voit.
Elle est peu valorisante pour l'entreprise car elle est souvent médiatisée.

Les grèves qui durent se finissent souvent mal.
Un \textit{lock out} est possible : occupation de l'entreprise, mais si on bloque l'accès à l'entreprise aux fournisseurs ou aux non grévistes c'est illégal.
Il y a parfois aussi séquestration des PDG.
C'est illégal, mais ça arrive.

La grève ne peut pas être un motif de licenciement.
Cela entrainerait une lourde sanctions pour l'employeur.
Dans la fonction publique il y a un préavis mais dans le droit privé il n'y a pas de préavis à donner.
L'entreprise ne peut pas créer des CDD pour remplacer des grévistes.
L'entreprise ne peut pas faire grand-chose donc elle devrait relancer les discussions.