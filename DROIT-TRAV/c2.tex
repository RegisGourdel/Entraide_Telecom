L'administration du travail possède une structure pyramidale à trois niveaux :

\subsection{Le ministère du travail}
	
	Créé en 1906 par Clemenceau, la direction générale du travail s'occupe :
	\begin{itemize}
		\item de l'élaboration des textes ;
		\item de l'application des textes : circulaires, comment lire et comprendre la loi ;
		\item des négociations collectives.
	\end{itemize}


\subsection{Direction déconcentrée de travail : les DIRECCTE}

	DIRECCTE est l'accronyme de Direction Régionale des Entreprises, de la Concurrence, de la Consommation, de Travail et de l'Emploi.
	
	Elles opèrent au niveau régional et dépendent du tribunal administratif (pas des prud'hommes) avec les fonctions suivantes :
	\begin{itemize}
		\item application des textes ;
		\item négociations collectives ;
		\item ruptures conventionnelles ;
		\item plans sociaux.
	\end{itemize}


\subsection{L'inspection du travail}

	Créée en 1874, l'inspection du travail, constituée de hauts fonctionnaires rattachés au minisère du travail, est coupée en secteurs de travail (agriculture, artisanat,...).
	Elle est le premier interlocuteur de l'employé (le deuxième étant les syndicats et le troisième un avocat).
	Elle opère au niveau départementale avec les tâches comme :
	\begin{itemize}
		\item constater des infractions ;
		\item faire fermer un chantier ;
		\item rédiger des PV.
	\end{itemize}
	Il y a un pouvoir de contrôle et de sanction mais aussi d'information : conseiller les salariés, rédaction de réclamation.
	
	Le réflexe doit être d'aller voir l'inspection du travail en cas de problème de conditons de travail, rémunération, etc...
