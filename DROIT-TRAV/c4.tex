Un contrat de travail est par défaut est à durée indéterminée.
On peut spécifier la durée (CDD).

Pour la suite, on parlera de CDI.

Pour le CDI il n’y a pas d’obligation d’avoir un contrat de travail écrit.
Un simple accord oral suffit.
On appelle cela un contrat verbal.
Si un contrat passé à l'oral est annulé aussi à l'oral avant la date d'embauche, il y a possibilité de saisir un tribunal pour rupture de franc parlé afin d'être remboursé de ses frais.

Pour le CDD, il faut obligatoirement un contrat écrit.
Cependant il est très recommandé de mettre en place un écrit.
Toujours penser à formaliser par écrit les accords passés (mail y compris).


\subsection{La promesse d'embauche}
	Une promesse d’embauche vaut un contrat de travail.
	Le non-respect d’une une promesse d’embauche équivaut à une rupture abusive, on peut alors demander dommages et intérêts même si l'on avait pas commencé à travailler.

	\paragraph{}
	Les points à faire figurer dans une promesse d’embauche sont :
	\begin{itemize}
		\item[\textbullet] Date d’engagement.
		\item[\textbullet] Rémunération.
		\item[\textbullet] Fonctions (description du poste).
		\item[\textbullet] Nombre d’heures (temps de travail).
		\item[\textbullet] Noms de l'employeur et du salarié.
		\item[\textbullet] Prévoyance mutuelle et n° de sécurité sociale.
	\end{itemize}
	
	La localisation du poste et la délégation de pouvoir peuvent également y être.
	L'employeur ne peut pas revenir sur ces points sans accord du salarié.
	La signature vaut accord.

\subsection{Les modalités d'embauche et le contrat}
	Toutes heures effectués au-dessus de 35 sont des heures supplémentaires.
	En dessous de 35 heures, c’est un contrat à temps partiel.
	Pour les employés itinérants ou cadres, on utilise les forfaits jours.
	On définit le nombre de jours travaillés.

	Pour un contrat à temps partiel (30h par exemple), au dela de 10\% d’heure supp, il y a majoration des heures travaillées.

	Si aucune rémunération n'est spécifiée, c’est un contrat de bénévolat...

	\paragraph{}
	La convention collective, détermine la classification des postes de travail, les minimums de salaires pour un poste donné.
	Le SMIC est le minimum salarial.
	La CC comporte des grilles de salaire selon le poste.
	Il n’est pas légal de payer un employé en dessous du salaire indiqué par la CC.

	\paragraph{Le contrat.}
	Il n’existe pas de définition stricte du contrat de travail.
	Par convention, le contrat de travail est la convention par laquelle une personne (le salarié) s’engage à travailler moyennant rémunération pour le compte d’une autre personne (l’employeur) sous la subordination de laquelle elle se place.

	\textrightarrow\ Lien de subordination : définition des contraintes et ordres que peut recevoir l’employé de la part d’un supérieur qui a aussi droit de contrôle.

	\paragraph{}
	Dans certains cas des prestations de travail bénévoles (issus de démarches volontaires) ont été requalifiées en contrat de travail en raison de ce lien de subordination.
	Exemple : bénévoles de la croix-rouge soumis à des plannings, ou candidats de télé-réalité soumis à un scénario avec peu d'autonomie.

\subsection{Les clauses du contrat de travail}

	Des clauses ajoutables au contrat sont :
	\begin{itemize}
		\item[\textbullet] Exclusivité
		\item[\textbullet] Période d’essai (facultative).
		\item[\textbullet] Confidentialité
		\item[\textbullet] Clause de non concurrence : 
		\item[\textbullet] Clause de mobilité.
		\item[\textbullet] Clause de dédi-formation : paiement sur le temps de travail d’une formation diplômante ou non, mais en contrepartie, l’employé s’engage à reste dans l’entreprise (2 à 5 ans). En cas de rupture par l'employé il devra rembourser les frais engagés.
		\item[\textbullet] Terme de travail
		\item[\textbullet] Contrat de couple : clause d'indivibilité dans le cas où deux conjoints sont embauchés par une entrerprise. Mais cette indivibilité doit être justifiée.
		\item[\textbullet] Clause de rupture
		\item[\textbullet] Test professionnel (quelques heures, test de capacité) ?
		\item[\textbullet] Tenue de travail.
		\item[\textbullet] Alcool ?
		\item[\textbullet] Clause de rupture (absence injustifiées, objectifs non atteints).
	\end{itemize}
	
	\paragraph{La période d'essai.}
	C'est un test de compétences et d'aptitudes pour l'entreprise, qui permet de vérifier que le travail convient à l'employé.
	Elle dépend du statut au sein de l’entreprise. 
	Le maximum est de 4 mois pour les cadres, 3 mois pour les techniciens et agents de maitrise, 2 mois tout autre employé.
	On peut renouveler la période d’essai pour la même durée sans justification.
	
	Pendant la période d’essai, l’employé peut annoncer une rupture 48h avant après 8 jours de présence ou 24h si la période est inférieure à 8 jours.
	Même chose de la part de l'entreprise jusqu'à un mois, puis 2 semaines après 1 mois de présence, 1 mois après 3 mois de présence.
	
	\paragraph{La clause de non-concurrence.}
	Elle est faite pour empêcher l’employé de travailler dans une entreprise concurrente pendant une certaine période.
	Pour que la clause soit valable, il faut spécifier une limitation dans le temps (maximum 2 ans) et dans l’espace, et il faut une contrepartie financière.
	Les deux conditions sont obligatoires pour que la clause soit valable.
	
	La limitation géographique doit être cohérente avec l'implantation de l'entreprise.
	Cela ne doit pas revenir à s'expatrier, mais plutôt concerner un département ou une région, mais au cas par cas sur la France.
	
	Il faut que la compensation financière ne soit pas dérisoire.
	Les montants classiques, sont donnés dans la convention collective (entre 30 et 50\% du salaire par mois selon les critères de la jurisprudence).
	On touche la compensation après la fin du contrat de travail.
	
	L’employeur peut faire appel à la formation de référé si l’employé ne respecte pas la clause de non concurrence.
	Cette clause ne dépend pas du type ou du motif de rupture.
	En revanche le type de fonction est un argument pris en compte.
	
	Il existe un délai pour délier de cette clause à la rupture du contrat : 15 jours environ.
	
	\paragraph{La clause de mobilité.}
	C'est une contrainte sur le déplacement géographique du lieu de travail (mutation), sans contrepartie financière.
	Elle vaut pour motif de rupture du contrat en cas de refus.
	Règle : être mobile sur toutes les agences.
	C'est un garde-fou pour empêcher l'utilisation disciplinaire.
	Elle doit être justifiée dans l'intérêt de l'entreprise et ne doit pas être un bouleversement de la vie familiale.
	En pratique on est souvent aidé pour trouver un logement, le déménagement est payé...
	
	\paragraph{La tenue de travail.}
	On peut contraindre les salariés à porter une tenue de travail, qui est différente de la tenue vestimentaire, et peut servir pour identifier les employés (différencier du consommateur).
	La nécessité de cette tenue doit être justifiée par l'intérêt de l'entreprise, sinon c'est une atteinte à la vie privée (pourquoi un tailleur jupe pour les femmes ?).
	Ce point est souvent précisé dans le réglement intérieur de l'entrerprise, son absence donne lieu à un avertissement.
	La tenue est fournie par l'entreprise.
	
	Se pose alors le problème de son entretien.
	Depuis un jugement de la cour de cassation en 2005, il est à la charge de l'employeur.
	Un service d'entretien est mis en place ou une compensation est prévue avec les partenaires sociaux.