\paragraph{Motif de CDD :} stage, Apprentissage, intérimaire (CDD indirect utilisant le CTT « contrat de travail temporaire », relation triangulaire), remplacement (maladie, congé maternité, accident du travail, congés), surcroît d’activité.

La durée maximale du CDD est de 18 mois, renouvelable deux fois.

Il contient les mêmes mentions que le CDI ainsi que la durée exacte de date à date.
S’il n’y a pas de contrat, ça devient un CDI (dans le cas des intérimaires, le salarié ne devient pas employé de l'entreprise d’intérim, mais directement de l'entreprise utilisatrice).

\paragraph{}
Dans le cas d’un remplacement pour arrêt maladie ou accident travail, on connait la date début mais pas celle de fin.
On emploie alors avec un contrat de travail à terme imprécis.
Ce dernier doit indiquer le nom du salarié remplacé et ses fonctions au sein de l’entreprise.
La rémunération n’est pas forcément la même que celle du salarié remplacé.
Mais le salarié CDD peut toucher absolument tous les avantages que touchait l’ancien employé qui était en CDI.
Il n'y a pas d’obligation pour l’employeur d’engager le remplaçant en cas de démission du salarié en congé.

À l’issue du CDD de remplacement est prévu une indemnité de précarité de l’ordre de 10\% de la rémunération totale durant toute la durée du contrat, heures sup comprise (sauf accord collectifs ou accord de branche avec un pourcentage inférieur).
Cette indemnité est annulée si l’employé remplaçant est embauché à l’issu du CDD en CDI.
Si l’on refuse le CDI, on perd également l’indemnité.
L’abandon de poste supprime aussi l’indemnité.

\paragraph{}
Si l’on rompt le CDD pour un CDI, on peut démissionner.
En cas de démission formelle, l’employeur peut demander au conseil des prud’hommes de condamner le salarié de payer les salaires restants avant la fin du contrat.
En cas de faute grave, l’employeur peut virer le salarié sans préjudice en retour.

\paragraph{}
Pour un arrêt de travail pour maladie il faut un document fourni par le médecin.
Deux feuillets : un pour la sécu, l’autre pour l’employeur.
L’employé a deux jours pour remettre le document à la sécu et l’employeur.
La raison médicale de l’absence n’est pas communiquée à l’employeur.

En cas de doute de la véracité de l’arrêt maladie, l’employeur peut demander une contre visite médicale pour vérifier l’arrêt maladie.
Le salarié ne peut pas refuser la consultation.
Une contre visite ne vaut que pour un arrêt.
Si le médecin traitant re-délivre un arrêt, il faut effectuer une 2\up{e} contre visite.

On peut licencier le salarié malade sous certaines conditions :
\begin{itemize}
	\item[\textbullet] Absence prolongées (durée indiquée dans la CC, 4 à 6 mois).
	\item[\textbullet] Perturbation au sein de l’entreprise.
	\item[\textbullet] Remplacement définitif du salarié absent.
\end{itemize}

Au retour d’un arrêt de travail supérieur à 1 mois, il faut revenir voir le médecin du travail sous 8 jours pour définir l’aptitude au travail.
Le médecin du travail peut donner une aptitude complète, partielle ou nulle.

Une deuxième visite au bout de 15 jours est possible si pas d’accord du médecin.
Durant cette deuxième visite, le médecin peut dire si l’employé est inapte à tous postes.
Dans ce cas, on peut licencier sous 1 mois le salarié, appelé mois de reclassement (non payé).