Pour licencier un employé, il faut au préalable le convoquer pour un « entretien préalable » par écrit (recommandé AR ou en main propre) pour laisser l’employé s’expliquer devant les faits qui lui sont reprochés.

\paragraph{La lettre de convocation à l'entretien préalable}

La lettre doit préciser l’heure et la date.

Si l’entreprise a plus de 11 salarié, on suppose qu’il y a un représentant du personnel.
Le salarié peut se présenter en se faisant assister par un autre salarié de l’entreprise (représentant du personnel par exemple, pas d’avocat).
Ou alors un conseiller du salarié, personne désignée par un syndicat auprès de la préfecture, consultable à la mairie de son domicile ou auprès de l’inspection du travail par exemple.
Il faut que la lettre de convocation comporte ces informations avec les adresses et tous les détails.

Si moins de 11 salarié, ce peut être juste un autre salarié ou conseiller du salarié.

Il faut aussi respecter certains délais.
Il doit s’écouler un délai de 5 jours ouvrable entre la réception (passage du facteur) de la lettre de convocation et le rendez-vous.
Mentionner « reçu le + signature de l’employé » si remise en main propre pour pouvoir dater la remise.

La lettre n’est pas obligée de mentionner les raisons du licenciement.

Dans la fonction publique, au contraire, le fonctionnaire peut demander accès et faire des copies de son dossier contenant par exemple le motif de la ou les fautes.


\paragraph{L'entretien}

Pendant l’entretien, l’employeur précise les reproches à l’employé.
L'entretien ne doit pas être informel.
L’employeur est censé ne pas déjà avoir pris sa décision.
Il n’a pas le droit de dire verbalement « je vous licencie », il faut dire « j’envisage ».
Si le licenciement porte sur la personne même du salarié (pas un licenciement économique) on parle de \textit{licenciement pour motif personnel}.

L’accompagnant sert ainsi de témoin à l’entretien.
L’employeur aussi peut se faire assister (pas d’avocat, une ou deux personnes de l'entreprise).
Lors de la fin de l’explication des faits, l’employé peut se défendre et la personne qui l'assiste peut poser des questions.
Il n'y a pas de durée à respecter.

À l’issue de l’entretien, soit l’employeur a été convaincu par la défense de l’employé et donc arrêter la procédure et abandonne la sanction disciplinaire / licenciement.
Soit il ne change pas d’avis, il doit alors communiquer par écrit recommandé AR le salarié sa décision.


\paragraph{La lettre de licenciement}

La lettre peut être envoyer par mail mais il faut pouvoir prouver que le salarié a bien reçu et lu le mail.
Un SMS n'est pas valable.
Il faut 2 jours ouvrables entre l’entretien et l’envoi du courrier.
Cela permet un temps de réflexion pour l’employeur.

Dans le cas d'un' licenciement économique, la lettre de licenciement ne peut pas être envoyée avant un délai de 7 jours ouvrables après l’entretien, ou 15 pour un cadre.

Le délai maximum est de 1 mois avant d’envoyer la lettre de sanction / licenciement.

La lettre de licenciement doit contenir précisément tous les motifs du licenciement.
Elle fixe les limites du litige et sert de base au conseil des prud’homme.
La lettre ne devrait pas ajouter de motifs supplémentaires à ceux données pendant l’entretien.
C’est tolérable mais « irrégulier ».

Si le licenciement n’est pas économique, il doit être basé sur une cause réelle et sérieuse.
L’employeur doit se baser sur des éléments objectifs.
Il doit donner avec précision ces éléments (dates, heure, personnes impliquées, lieux).
Il n’existe pas de liste officielle des motifs de licenciement.

L’insuffisance professionnelle constitue un motif de licenciement en soit à condition de le justifier par des faits.

Pour les plus anciens employés, on ne peut pas remonter trop loin dans le temps pour les motifs.
Le maximum est de deux mois entre le fait et la constatation du fait par l’employeur.
La jurisprudence réduit encore ce délai.
Les faits anciens permettent d'expliquer le contexte mais ne sont pas une justification.


\paragraph{Le préavis.}
Lors d'un licenciement, la rupture du contrat n'intervient pas immédiatement après sa notification.
Un délai entre la notification du licenciement et la fin du contrat de travail doit être respecté, qui constitue le préavis.

L’employeur peut demander à l’employé d’effectuer son préavis.
Il peut aussi l’en dispenser : le salarié ne vient plus au travail mais il faudra payer les mois de salaires quand même.
On peut engager un licenciement pour faute lourde si pendant le préavis l’employé devient désagréable et pollue volontairement l’ambiance de travail pour être dispensé du préavis.

Pendant le préavis, les congés payés reportent le préavis du montant de leur durée.

Si l’employé trouve un emploi pendant un préavis dispensé, il peut être embauché et toucher son salaire plus le préavis.


\paragraph{Les indemnités de licenciement.}
L'indemnité de licenciement est calculée à partir de la rémunération brute perçue par le salarié avant la rupture de son contrat de travail.

L'indemnité légale ne peut pas être inférieure à 1/5\up{e} d'un mois de salaire multiplié par le nombre d'années d'ancienneté. Au-delà de 10 ans d'ancienneté, il faut y ajouter 2/15\up{e} d'un mois de salaire par année supplémentaire.
Le calcul de l'indemnité prend également en compte l'année en cours.

En cas de licenciement durant un long arrêt maladie simple, on touche l’indemnité de licenciement classique, pas de préavis.


\paragraph{Autres droits lors du départ.}
Le salarié a aussi le droit à ses congés payés ainsi qu'au 13\up{e} mois au pro rata de ce qu’il a effectué en terme de temps de travail dans l’année.
On appelle cela un solde tout compte.
Une démission garantie quand même le règlement du solde tout compte.\\

Pour la mutuelle, est rendue obligatoire la mise en place d’une mutuelle collective (une partie sur le salaire, une partie payée par l’entreprise).
Le salarié peut demander la portabilité (9 à 12 mois) de la mutuelle.
Il en profite donc même après la fin du contrat.\\

Doit être remis au salarié le reçu pour solde de tout compte et le certificat de travail spécifiant la date d’emploi et la fonction.
On ne peut y mentionner les causes de la fin du contrat.
Est aussi remise une attestation pôle emploi pour toucher les allocations chômage.\\

Dans le cas où le licenciement se fait après un accident du travail, une maladie grave ou un problème d'inaptitude au travail, l'indemnité de licenciement est doublée et le préavis est payé même s'il n'est pas effectué.
Si le licenciement se fait durant un long arrêt maladie simple, on touche l'indemnité de licenciement classique et il n'y a pas d'indemnité pour la période de préavis qui n'a pas pu être effectuée.

Le délai de carence (temps minimum avant d'être pris en charge) pour pôle emploi est de septs jours auquels il faut ajouter la durée des congés payés et le préavis.

On touche au moins 57\% de l'ancien salaire pendant deux ans au chômage si la rupture est à l’origine de l’employeur.
Elle est de 70\% en cas de motif économique (au maximum, i.e. sur une première période).
La démission supprime les allocations chômage.

La meilleure stratégie pour l’employé, c’est de ne pas démissionner.
Faire un abandon de poste : ne pas justifier les absences, ne pas répondre aux appels et lettres de l’entreprise.
On force donc le licenciement pour faute grave.

Le droit présente des incohérences : des aides pour la création d'entreprise sont réservées à ceux qui se sont fait licencier, et inaccessibles à ceux qui ont démissionné.


\paragraph{Fautes grave et lourde.}
En cas de faute grave, le temps de prescription se réduit grandement.
Il existe également la faute lourde qui est pire que la faute grave.
Aujourd’hui, faute grave et lourde sont presque pareilles en termes de sanction financière.
On touche quand même allocation chômage et congés.

Différence entre faute lourde et grave : en cas de faute lourde (atteinte volontaire et préméditée à la pérennité de l'entreprise) l’employeur peut saisir le conseil des prud’homme pour dommages et intérêts contre l’employé.

En cas de faute grave il n'y a pas de préavis ni d’indemnité de licenciement. Le salarié touchera juste le solde tout compte (congés et 13\up{e} mois).
L’employeur peut le mettre à pied à titre conservatoire dès la convocation à l’entretien.
Il y a toujours obligation de remettre l'attestation pôle emploi (chômage) et l' attestation de travail.
On a toujours le droit au chômage.


\paragraph{La faute simple.}
C'est le cas si les objectifs ne sont pas atteints ou qu'une erreur est répétée en dépit de la formation.
Ici, on a droit au préavis et indemnité de licenciement (plus d’un an d’ancienneté dans l’entreprise pour indemnité de licenciement).
Le préavis dépend de l’ancienneté et du poste (cadre ou non cadre), voir convention collective.


\paragraph{}
Il existe d’autres causes de fin de contrat, dont par exemple la rupture conventionnelle (accord commun entre salarié et entreprise, consentement mutuel entre les deux partis).
Cela évite la procédure de licenciement et donne droit à des indemnité de licenciement (appelée ici indemnité spécifique de rupture, même calcul que normalement).
Il y a toujours droit aux allocations chômage.
