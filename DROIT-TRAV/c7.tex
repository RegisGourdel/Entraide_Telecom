Pendant toute l’exécution du contrat de travail, le salarié peut demander pendant ou en dehors des heures de travail une formation.
Cette formation doit être en relation avec le poste effectué.

Auparavant nommé DIF (droit individulel à la formation), cela s’appelle maintenant le compte personnel de formation (CPF).
Le CPF permet d’acquérir des  droits à la formation professionnelle (heures CPF).
Ces droits sont inscrits dans un compte d’heures qui suit le salarié tout au long de sa vie professionnelle.
Il peut décider de se former régulièrement en les utilisant.\\

Toutes les personnes de 16 ans et plus bénéficient d’un compte personnel de formation, jusqu’à ce qu’elles aient fait valoir l’ensemble de leurs droits à la retraite.
Pour les salariés de droit privé le compte est crédité automatiquement en heures sur la base des déclarations de l'employeur.

À temps complet, un compte est alimenté à raison de 24 heures par an jusqu’à 120 heures, puis de 12 heures par an jusqu’à un maximum de 150 heures.