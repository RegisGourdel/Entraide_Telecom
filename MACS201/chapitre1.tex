\begin{defn}
	Let $\mathcal{H}$ be a complex linear space.
	An \textbf{inner-product} on $\mathcal{H}$ is a function $\scal{\cdot}{\cdot} \colon \mathcal{H} \times \mathcal{H} \to \C$ which satisfies the following properties :
	\begin{enumerate}[(i)]
		\item $\forall (x,y) \in \mathcal{H} \times \mathcal{H}, \scal{x}{y} = \overline{\scal{y}{x}}$,
		\item $\forall x, y, z \in \mathcal{H} \forall (\alpha,\beta) \in \C \times \C, \scal{\alpha x + \beta y}{z} = \alpha \scal{x}{z} + \beta \scal{y}{z}$,
		\item $\forall x \in \mathcal{H}, \left( \scal{x}{x} = 0 \right) \iff \left( x = 0 \right)$
	\end{enumerate}
	Then $\norme{\cdot} \colon x \mapsto \sqrt{\scal{x}{x}} \geq 0$ defines a norm on $\mathcal{H}$.
	Both are continuous.
\end{defn}

\begin{thm}
	For all $x, y \in \mathcal{H}$, we have :
	\begin{enumerate}[a)]
		\item Cauchy-Schwarz inequality : $\abs{\scal{x}{y}} \leq \norme{x} \cdot \norme{y}$,
		\item triangular inequality : $\abs{\norme{x} - \norme{y}} \leq \norme{x - y} \leq \norme{x} + \norme{y}$,
		\item Parallelogram inequality : $\norme{x + y}^2 + \norme{x - y}^2 = 2 \norme{x}^2 + 2 \norme{y}^2$.
	\end{enumerate}
\end{thm}

\begin{defn}
	An inner-product space $\mathcal{H}$ is called an Hilbert space if it is complete.
\end{defn}

\begin{pop}
	For all measured space $(\Omega, \mathcal{F}, \mu)$, the space $L^2(\Omega, \mathcal{F}, \mu)$ endowed with $\scal{f}{g} = \int f \bar{g} \diff \mu$ is a Hilbert space.
\end{pop}

\begin{defn}
	Two vectors $x,y \in \mathcal{H}$ are orthogonal if $\scal{x}{y} = 0$ which we denoted by $x \perp y$.
	If $\mathcal{S}$ is a subspace of $\mathcal{H}$, we write $x \perp \mathcal{S}$ if $\forall s \in \mathcal{S}, x \perp s$.
	Also we write $\mathcal{S} \perp \mathcal{T}$ if all vectors in $\mathcal{S}$ are orthogonal to $\mathcal{T}$.
\end{defn}

\begin{note}
	If $\mathcal{H} = \mathcal{A} + \mathcal{B}$ and $\mathcal{A} \perp \mathcal{B}$ we will denote $\mathcal{H} = \mathcal{A} \overset{\perp}{\oplus} \mathcal{B}$.
\end{note}

\begin{defn}
	Let $\mathcal{E}$ be a subset of an Hilbert space $\mathcal{H}$.
	The orthogonal set of $\mathcal{E}$ is defined as $\mathcal{E} = \{ x \in \mathcal{H} \mid \forall y \in \mathcal{E}, \scal{x}{y} = 0 \}$.
\end{defn}

\begin{thm}
	If $\mathcal{E}$ is a subset of an Hilbert space $\mathcal{H}$, then $\mathcal{E}^{\perp}$ is closed.
\end{thm}

\begin{defn}
	Let $E$ be a subset of $\mathcal{H}$.
	It is an orthogonal set if for all $(x,y) \in E \times E, x \neq y, x \perp y$.
	If moreover $\forall x \in E, \norme{x} = 1$, we say that $E$ is orthonormal.
\end{defn}

\begin{thm}
	Let $(e_{i})_{i \geq 1}$ be an orthonormal sequence of an Hilbert space $\mathcal{H}$ and let $(\alpha_{i})_{i \geq 1} \in \C^{\N}$.
	The series $\sum_{i = 1}^\infty \alpha_i e_i$ converges in $\mathcal{H}$ if and only if $\sum_i \abs{\alpha_i}^2 < \infty$, in which case $\norme{\sum_{i = 1}^\infty \alpha_i e_i}^2 = \sum_{i = 1}^\infty \abs{\alpha_i}^2$.
\end{thm}

\begin{pop}
	Let $x \in \mathcal{H}$ (Hilbert space) and $E = \{ e_1, \ldots, e_n \}$ a finite orthonormal set of vectors.
	Then $\norm{x - \sum_{k = 1}^n \scal{x}{e_k} e_k}^2
		= \norm{x}^2 - \sum_{k = 1}^n \abs{\scal{x}{e_k}}^2
		= \inf \{ \norm{x - y}^2, y \in \Span(e_1, \ldots, e_n) \}$.
\end{pop}

\begin{cor}[Bessel inequality]
	Let $(e_i)_{i \geq 1}$ be an orthonormal sequence of a Hilbert space $\mathcal{H}$.
	Then$\forall, x \in \mathcal{H}, \sum_{i = 1}^\infty \abs{\scal{x}{e_i}}^2 \leq \norm{x}^2$.
\end{cor}

\begin{defn}
	A subset $E$ of a Hilbert space $\mathcal{H}$ is said dense id $\overline{\Span}(E) = \mathcal{H}$.
	An orthonormal dense sequence is called a Hilbert basis.
\end{defn}

\begin{pop}
	Consider the measured space $(\Omega, \mathcal{F}, \mu)$ and the Hilbert space $\mathcal{H} = L^2(\Omega, \mathcal{F}, \mu)$, $\overline{\Span}(\indic_A, A \in \mathcal{F}) = \mathcal{H}$.
\end{pop}

\begin{thm}
	Let $(e_i)_{i \geq 1}$ be a Hilbert basis of the Hilbert space $\mathcal{H}$.
	Then $\forall x \in \mathcal{H}, x = \sum_{i = 1}^\infty \scal{x}{e_i} e_i$.
\end{thm}

\begin{thm}
	Let $(e_i)_{i \geq 1}$ be an orthonormal sequence of the Hilbert space $\mathcal{H}$.
	The following assertions are equivalent :
	\begin{enumerate}[(i)]
		\item $(e_i)_{i \geq 1}$ is a Hilbert basis,
		\item if some $x \in \mathcal{H}$ satisfies $\forall i \geq 1, \scal{x}{e_i} = 0$ then $x = 0$,
		\item $\forall x \in \mathcal{H}, \norm{x}^2 = \sum_{i = 1}^\infty \abs{\scal{x}{e_i}}^2$.
	\end{enumerate}
\end{thm}

\begin{thm}
	A Hilbert space $\mathcal{H}$ if separable (i.e. contains a countable dense subset) if and only if it admits a Hilbert basis.
\end{thm}

\subsection{Fourier series}

	Let $\psi_n \colon x \mapsto \frac{1}{\sqrt{2\pi}} e^{inx}, n \in \Z$.
	Let $L^1(\T)$ denote the set of $2\pi$-periodic locally integrable functions.
	For $f \in L^1(\T)$, set $\forall n \in \N, f_n = \sum_{k = -n}^n \left( \int_\T f \bar\phi_k \right) \phi_k$.

	\begin{thm}
		Suppose that $f$ is a continuous $2\pi$-periodic function.
		Then the Cesaro sequence $\frac{1}{n} \sum_{k = 0}^{n - 1} f_k$ converges uniformly to $f$.
	\end{thm}

	\begin{cor}
		Let $\mu$ be a finite measure on the Borel sets of $\T = \R / (2\pi\Z)$.
		The sequence $(\phi_n)_{n \in \Z}$ is dense in the Hilbert space $L^2(\T, \mathcal{B}(\T), \mu)$.
	\end{cor}

	\begin{cor}
		The sequence $(\phi_n)_{n \in \Z}$ is a Hilbert basis in $L^2(\T)$.
		In particular, $\forall f \in L^2(\T), f = \sum_{k = -\infty}^\infty \alpha_k \phi_k$ with $\alpha_k = \frac{1}{\sqrt{2\pi}} \int_\T f(x) e^{-ikx} \diff x$ when the infinite sum converges in $L^2(\T)$.
		The Parseval identity then reads $\int_\T \abs{f(x)}^2 \diff x = \sum_{k = -\infty}^\infty \abs{\alpha_k}^2$.
	\end{cor}