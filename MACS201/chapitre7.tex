\subsection{Invariant measures and stationarity}

	Let $P$ be a Markov kernel on $(\mathsf{X},\mathcal{X})$.

	\begin{defn}
		A non zero $\sigma$-finite positive measure $\mu \in \mes_+(\mathsf{X},\mathcal{X})$ is said to be \textbf{invariant} with respect to $P$ (or $P$-invariant) if $\mu = \mu P$.
	\end{defn}

	\begin{thm}
		A Markov chain $(X_k)_{k \in \N}$ defined on $(\Omega,\mathcal{F},\proba)$ with kernel $P$ is a stationary process iff its initial distribution is invariant with respect to $P$.
	\end{thm}

	\begin{defn}
		A non-empty set $B \in \mathcal{X}$ is called \textbf{absorbing} if $\forall x \in B, P(x,B) = 1$.
	\end{defn}

	\begin{thm}
		\begin{enumerate}[(i)]
			\item The set of invariant probability measures for $P$ is a convex subset of the convex cone $\mes_+(\mathsf{X},\mathcal{X})$.
			\item Let $\pi$ be an invariant probability and $\mathsf{X}_1 \subset \mathsf{X}$ with $\pi(\mathsf{X}_1) = 1$.
				There exists $B \subset \mathsf{X}_1$ such that $\pi(B) = 1$ and $B$ is absorbing for $P$.
		\end{enumerate}
	\end{thm}


\subsection{Stopping times}

	\begin{defn}
		A random variable $\tau \colon \Omega \to \Bar{\N} = \N \cup \{ \infty \}$ is called a \textbf{stopping time} if $\forall k \in \N, \{ \tau \leq k \} \in \mathcal{F}_k$.
		The family $\mathcal{F}_\tau$ of events $A$ such that $\forall k \in \N, A \cap \{ \tau \leq k \} \in \mathcal{F}_k$ is called the $\sigma$-field of events prior to times $\tau$.
	\end{defn}

	\begin{defn}
		For $A \in \mathcal{X}$, the \textbf{first hitting time} of the set $A$ by the process $(X_n)_{n \in \N}$ is $\tau_A = \inf \{ n \geq 0 \mid X_n \in A \}$ and \textbf{return time} is $\sigma_A = \inf \{ n \geq 1 \mid X_n \in A \}$, where, by convention, $\inf \emptyset = + \infty$.
		The successive return times $\sigma_A^{(n)}$ are defined successively by $\sigma_A^{(0)} = 0$ and $\forall k \geq 0, \sigma_A^{(k + 1)} = \inf \left\{ n > \sigma_A^{(n)} \mid X_n \in A \right\}$.
	\end{defn}

	\begin{pop}
		Let $(\Omega, \mathcal{F}, (\mathcal{F}_k)_{k \in \N}, \proba)$ be a filtered space and $\tau$ and $\sigma$ be two stopping times for $(\mathcal{F}_k)_{k \in \N}$.
		Denote by $\mathcal{F}_\tau$ and $\mathcal{F}_\sigma$ the $\sigma$-fields of the events prior to $\tau$ and $\sigma$ respectively.
		Then,
		\begin{enumerate}[(i)]
			\item $\tau \wedge \sigma$, $\tau \vee \sigma$ and $\tau + \sigma$ are stopping times,
			\item if $\tau \leq \sigma$, then $\mathcal{F}_\tau \subset \mathcal{F}_\sigma$,
			\item $\mathcal{F}_{\tau \wedge \sigma} = \mathcal{F}_\tau \cap \mathcal{F}_\sigma$,
			\item $\{ \tau < \sigma \} \in \mathcal{F}_\tau \cap \mathcal{F}_\sigma$ and $\{ \tau = \sigma \} \in \mathcal{F}_\tau \cap \mathcal{F}_\sigma$.
		\end{enumerate}
	\end{pop}
