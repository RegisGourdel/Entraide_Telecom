\subsection{Calcul sur les événements}

	\begin{pop}
		\begin{itemize}
			\item[\textbullet] Si $(A_n)_n$ est croissante, $\proba(\bigcup_n A_n) = \lim_{n \to \infty} \proba(A_n)$.
			\item[\textbullet] Si $(A_n)_n$ est décroissante, $\proba(\bigcap_n A_n) = \lim_{n \to \infty} \proba(A_n)$.
			\item[\textbullet] Si $\forall n, \proba(A_n) = 0$ alors $\proba \left( \bigcup_n A_n \right) = 0$.
			\item[\textbullet] Si $\forall n, \proba(A_n) = A$ alors $\proba \left( \bigcap_n A_n \right) = 1$.
		\end{itemize}
	\end{pop}
	
	\begin{defn}
		$\limsup_{n \to \infty} A_n = \bigcap_{n \in \N} \bigcup_{k \geq n} A_k$, i.e. $\omega \in \limsup_n A_n \iff \forall n, \exists k \geq n, \omega \in A_k$.
	\end{defn}
	
	Donc $\limsup_n A_n$ est réalisé ssi une infinité de $A_n$ est réalisé.
	
	\begin{lem}[de \textbf{Borel-Cantelli}]
		Si $\sum_n \proba(A_n) < \infty$, alors $\proba(\limsup_n A_n) = 0$.
	\end{lem}

	Autrement dit, il y a une proba 1 pour que seulement un nombre fini de $A_n$ soient réalisés.


\subsection{Convergence p.s., en probabilité et dans $L^p$}

	\begin{defn}
		\begin{enumerate}[(i)]
			\item On dit que $X_n \overset{\proba}{\longrightarrow} X$ (\textbf{converge en probabilité}) si $\forall \epsilon > 0, \proba(\norme{X_n - X} > \epsilon) \underset{n \to \infty}{\longrightarrow} 0$.
			\item On dit que $X_n \overset{\text{p.s.}}{\longrightarrow} X$ (\textbf{converge presque sûrement}), si $\forall \omega\ \proba\text{-p.p}, X_n(\omega) \to X(\omega)$.
				Autrement dit il existe $A \in \mathcal{F}$ tel que $\proba(A) = 1$ et $\forall \omega \in A, \lim_n X_n(\omega) = X(\omega)$.
			\item On dit que $X_n \overset{L^p}{\longrightarrow} X$ (\textbf{converge vers $X$ dans $L^p(\Omega, \R^d)$}) si $X_n, X \in L^p$ et $\esp \left( \norme{X_n - X}^p \right) \underset{n \to \infty}{\longrightarrow} 0$. 
		\end{enumerate}
	\end{defn}

	\begin{pop}
		On note $X_n = \left( X_n^{(1)},\ldots,X_n^{(d)} \right)$ sur $\mathcal{X} = \R^d$.
		Alors $X_n \overset{\text{p.s.}}{\longrightarrow} X$ p.s. (resp. en probabilité, dans $L^p$) ssi $\forall k \in \iniff{1}{d}, X_n^{(k)} \overset{\text{p.s.}}{\longrightarrow} X^{(k)}$ (resp. en probabilité, dans $L^p$).
	\end{pop}

	\begin{pop}
		Si $X_n \overset{\text{p.s.}}{\longrightarrow} X$ ou $X_n \overset{L^p}{\longrightarrow} X$ alors $X_n \overset{\proba}{\longrightarrow} X$.
	\end{pop}

	\begin{pop}
		Si $\forall \epsilon > 0, \sum_n \proba(\norme{X_n - X} > \epsilon) < \infty$ alors $X_n \overset{\text{p.s.}}{\longrightarrow} X$.
	\end{pop}

	\begin{pop}
		$X_n \overset{\proba}{\longrightarrow} X$ ssi de toute sous-suite $X_{\varphi(n)}$ on peut extraire une autre sous-suite $X_{\varphi \circ \psi(n)}$ telle que $X_{\varphi \circ \psi(n)} \overset{\text{p.s.}}{\longrightarrow} X$.
	\end{pop}

	\begin{thm}[de \textbf{continuité}]
		% $X_n, X$ v.a. sur $\R^d$.
		Soit $h \colon \R^d \to \R^p$ mesurable et continue sur $C \in \mathcal{B}(\R^d)$ tel que $\proba(X \in C) = 1$ :
		\begin{enumerate}[(i)]
			\item Si $X_n \overset{\text{p.s.}}{\longrightarrow} X$ alors $h \circ X_n \overset{\text{p.s.}}{\longrightarrow} h \circ X$
			\item Si $X_n \overset{\proba}{\longrightarrow} X$ alors $h \circ X_n \overset{\proba}{\longrightarrow} h \circ X$.
		\end{enumerate}
	\end{thm}

	\begin{thm}[\textbf{Loi forte des grands nombres}]
		Soit $(X_n)$ i.i.d. telle que $\esp(\norme{X_1}) < \infty$.
		Alors $\frac{1}{n} \sum_{i = 1}^n X_i \overset{\text{p.s.}}{\longrightarrow} \esp(X_1)$.
	\end{thm}

	\begin{thm}[\textbf{Loi faible des grands nombres}]
		Soit $(X_n)$ i.i.d. telle que $\esp(\norme{X_1}^2) < \infty$.
		On a $\frac{1}{n} \sum_{i = 1}^n X_i \overset{\proba}{\longrightarrow} \esp(X_1)$.
	\end{thm}
