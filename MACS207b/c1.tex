\subsection{La loi gaussienne scalaire}

	\begin{defn}
		Une v.a. $X$ sur $\R$ est dite \textbf{gaussienne standard} si sa loi de probabilité admet la densité $f(x) = \frac{1}{\sqrt{2 \pi}} \exp \left( - \frac{x^2}{2} \right)$.
	\end{defn}
	
	\begin{defn}
		Soit $\sigma \in \R_+$ et $m \in \R$.
		On dit que la v.a. réelle $Y$ suit la loi gaussienne $\mathcal{N}(m,\sigma^2)$ si $Y = \sigma X + m$ où $X$ suit la loi gaussienne standard.
	\end{defn}
	
	\begin{pop}
		Soit $X \sim \mathcal{N}(0,1)$.
		Sa transformée de Laplace est $\psi(z) = \esp \exp(zX) = \exp \left( \frac{z^2}{2} \right)$.
	\end{pop}
	
	\begin{pop}
		$Y \sim \mathcal{N}(m,\sigma^2)$ 
	\end{pop}