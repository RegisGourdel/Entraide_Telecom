\begin{defn}
	Soit le polynôme $P(\lambda) = \lambda^n + a_1 \lambda^{n-1} + \ldots + a_{n-1} \lambda + a_n$.
	Est dite « compagne du polynôme $P$ » la matrice suivante :
	$$C(P) = \begin{pmatrix}
		-a_1 & -a_2 & -a_3   & \ldots & \ldots & -a_{n-1} & -a_n \\
		1    & 0    &        &        &        &          &      \\
		0    & 1    & 0      &        &        &          &      \\
		     & 0    & 1      & \ddots &        &          &      \\
		     &      & \ddots & \ddots & \ddots &          &      \\
		     &      &        & 0      & 1      & 0        &      \\
		     &      &        &        & 0      & 1        & 0    \end{pmatrix}$$
\end{defn}

\begin{pop}
	Le polynôme caractéristique de $C(P)$ vaut $(-1)^n P(\lambda)$.
	La matrice a donc pour valeurs propres les racines de $P$.
\end{pop}

Ce lien prouve, par le théorème d'Abel, que la recherche des valeurs propres d'une matrice ne peut se faire en un nombre fini d'opérations au-delà de la dimension 5.

\paragraph{Méthode de Jacobi}
Soit $A \in \Sym^+_n(\R)$ non diagonale.
On dispose de $p$ et $q$, $p < q$ tels que $a_{pq} \neq 0$.
On définit la matrice suivante :
$$\Omega = I_n + \sin(\theta) \cdot (E_{pq} - E_{qp}) + (1 - \cos(\theta)) \cdot (E_{pp} + E_{qq})\ .$$
C'est la matrice de rotation d'angle $-\theta$ dans le plan défini par les $p$\up{e} et $q$\up{e} vecteurs de la base (donc orthogonale).
On pose $B = \transp{\Omega} A \Omega \in \Sym_n(\R)$, $c = \cos(\theta)$, $s = \sin(\theta)$ et $t = \tan(\theta)$.
On a alors\\
$\left\{ \begin{array}{ll}
	b_{ij} = b_{ji} = a_{ij} & \text{si } i \not\in \{ p, q \} \text{ et } j \notin \{ p, q \} \\
	b_{pi} = b_{ip} = c \cdot a_{pi} - s \cdot a_{qi} & \text{si } i \not\in \{ p, q \} \\
	b_{qi} = b_{iq} = s \cdot a_{pi} + c \cdot a_{qi} & \text{si } i \not\in \{ p, q \} \\
	b_{pp} = a_{pp} - t \cdot  a_{pq} \\
	b_{qq} = a_{qq} + t \cdot a_{pq} \end{array}
\right.$.

\begin{thm}
	Soit $A \in \Sym^+_n(\R)$ et $B$ la matrice obtenue par la construction précédente.
	On a alors les relations
	$$\sum_{i = 1}^n \sum_{j = 1}^n a_{ij}^2 = \sum_{i = 1}^n \sum_{j = 1}^n b_{ij}^2
	\qquad \sum_{i = 1}^n a_{ii}^2 + 2a_{pq}^2 = \sum_{i = 1}^n b_{ii}^2$$
	(le poids de la matrice se déporte sur la diagonale).
\end{thm}

\begin{thm}
	La suite des matrices obtenues par la méthode de Jacobi est convergente et converge vers une matrice diagonale contenant les valeurs propres de $A$.
\end{thm}

Pour accélerer la convergence il vaut mieux prendre $\abs{a_{pq}}$ maximum.

\begin{thm}
	Si toutes les valeurs propres de $A$ sont distinctes, alors la suite des produits des matrices $\Omega$ converge vers une matrice orthogonale dont les vecteurs colonnes constituent unn ensemble orthnormal de vecteurs propres de $A$.
\end{thm}
