\begin{defn}
	Soit le polynôme $P(\lambda) = \lambda^n + a_1 \lambda^{n-1} + \ldots + a_{n-1} \lambda + a_n$.
	Est dite « compagne du polynôme $P$ » la matrice suivante :
	$$C(P) = \begin{pmatrix}
		-a_1 & -a_2 & -a_3   & \ldots & \ldots & -a_{n-1} & -a_n \\
		1    & 0    &        &        &        &          &      \\
		0    & 1    & 0      &        &        &          &      \\
		     & 0    & 1      & \ddots &        &          &      \\
		     &      & \ddots & \ddots & \ddots &          &      \\
		     &      &        & 0      & 1      & 0        &      \\
		     &      &        &        & 0      & 1        & 0    \end{pmatrix}$$
\end{defn}

\begin{pop}
	Le polynôme caractéristique de $C(P)$ vaut $(-1)^n P(\lambda)$.
	La matrice a donc pour valeurs propres les racines de $P$.
\end{pop}

Ce lien prouve, par le théorème d'Abel, que la recherche des valeurs propres d'une matrice ne peut se faire en un nombre fini d'opérations au-delà de la dimension 5.