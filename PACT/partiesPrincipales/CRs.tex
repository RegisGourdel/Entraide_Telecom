\section{Descriptif chronologique des réunions}

	Les rendez-vous avec experts par modules ne sont pas répertoriés.

	Légende :
	\begin{itemize}
		\item \textcolor{Blue}{séance encadrée prévue dans l'emploi du temps}
		\item \textcolor{Purple}{séance en autonomie prévue dans l'emploi du temps}
		\item \textcolor{Green}{réunion non prévue dans l'emplois du temps mais encadrée}
		\item \textcolor{Red}{réunion non prévues dans l'emploi du temps en autonomie}
	\end{itemize}

	\begin{description}
		\item[\textcolor{Blue}{29.09.15}]
			Premier contact tuteurs : team building ( réalisation de tours en papier) et recherche d'idées de sujets pour les 2 thèmes PACT annoncés (CR en \ref{CR1}).
		\item[\textcolor{Red}{01.10.15}]
			Choix parmi 14 de 3 sujets : Jogging, SmartMouth et Délécom retenus.
		\item[\textcolor{Blue}{05.10.15}]
			Remise en question du sujet jogging, remplacé par le sujet Feux de Forêts.\\
			Présentation des sujets à l'instar d'une start-up cherchant des clients et des fonds, questions entre membres du groupe.\\
			1er tri des cartes modules.
		\item[\textcolor{Red}{07.10.15}]
			Discussion sur les détails des 3 projets retenus.\\
			Répartition du travail pour les posters et brouillons des posters pour la foire aux experts.
		\item[\textcolor{Blue}{12.10.15}]
			Foire aux experts : SmartMouth et 3Délécom abandonnés, nouvelle idée FeelList et Feux de forêts réalisable mais moins attrayant (CR en \ref{CR2}).
		\item[\textcolor{Blue}{14.10.15}]
			Foire aux modules puis débrief sur la foire aux experts avec tuteurs (un compte-rendu individuel effectué par présentation, voir \ref{CR3.1} à \ref{CR3.5}).
		\item[\textcolor{Red}{16.10.15}]
			Discussion du détail des 2 sujets encore en lice (FeelList et Feux de forêts).\\
			Classement des cartes modules.
		\item[\textcolor{Blue}{19.10.15}]
			Pour choisir entre FeelList et Feux de forêts tous les memebres qui aiment le moins un sujet doivent défendre ses avantages.\\
			Commencement de répartition des modules au tableau.\\
			Explications relatives au scénario utilisateur SES à rédiger pendant les vacances.
		\item[\textcolor{Red}{23.10.15}]
			Discussion détaillée sur FeelList (sujet principal).\\
			Scénario de secours Feux de forêts.\\
			(CR en \ref{CR4.1}, \ref{CR4.2} et \ref{CR4.3})
		\item[\textcolor{Blue}{05.11.15}]
			Scénario corrigé par experte ses, et modifications pour tenir compte de ses conseils.
		\item[\textcolor{Blue}{09.11.15}]
			Foire libre : mini-cours de classification et rendez-vous SES.
		\item[\textcolor{Red}{25.10.15 au soir}]
			Réunion pour prendre en compte les remarques faites au cours du rendez-vous SES.
		\item[\textcolor{Green}{13.11.15}]
			Point avec tuteurs (rapport PAN 1, contrats avec experts).\\
			Répartition des modules entre nous.\\
			(CR en \ref{CR6})
		\item[\textcolor{Blue}{16.11.15}]
			Cours de l'expert GL (présentation des diagrammes d'architecture et d'activité).
		\item[\textcolor{Purple}{17.11.15}]
			Réalisation des diagrammes.
			Envoi de mails aux experts pour prise de rendez-vous modules.
		\item[\textcolor{Red}{20.11.15}]
			Détails supplémentaires quant au fonctionnement de FeelList, au rôle exact de chaque module, à la prise de contacts avec experts, aux questions subsistant (CR en \ref{CR7}).
		\item[\textcolor{Blue}{24.11.15}]
			Séance avec l'expert GL : correction des diagrammes et création d'un dépôt git pour notre groupe.
		\item[\textcolor{Purple}{25.11.15}]
			Préparation du PAN 1 : répartition des tâches.\\
			Rendez-vous experts.\\
			(CR en \ref{CR8})
		\item[\textcolor{Red}{30.11.15}]
			Touches finales du rapport PAN 1.\\
			Préparation de l'intervention orale PAN 1.
		\item[\textcolor{Blue}{08.12.15}]
			Réunion avec tuteurs pour faire le point sur le PAN 1 et discuter de l'avancement jusqu'au PAN 2.
		\item[\textcolor{Red}{14.12.15}]
			Point en autonomie du groupe.
		\item[\textcolor{Blue}{15.12.15}]
			Plans de test avec l'expert GL.
		\item[\textcolor{Red}{04.01.16}]
			Point en autonomie pour organiser le focus group.
		\item[\textcolor{Red}{11.01.16}]
			Point en autonomie en prévision du PAN 2, distribution des tâches, organisation du rapport écrit et de la soutenance orale.
	\end{description}


\section{Compte-rendu du xx.xx.16 : 1\up{er} contact tuteurs}
	\label{CR1}


\section{Compte-rendu du 12.10.15 : Foire aux experts}
	\label{CR2}


\section{CR foire aux modules : SES}
	\label{CR3.1}


\section{CR foire aux modules : Electronique}
	\label{CR3.2}

% ...