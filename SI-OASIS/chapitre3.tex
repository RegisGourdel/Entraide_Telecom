\begin{defn}[Causalité] Vocabulaire : \\
\begin{itemize}
\item h est causale si $$\forall n<0, h_{n}=0$$
\item Un SLI est causal si sa réponse impulsionnelle est causale
\item RIF : un SLI à réponse impulsionnelle finie
\item RII : un SLI à réponse impulsionnelle infinie
\end{itemize}
\end{defn}

\begin{pop} .\\
\begin{itemize}
\item La convolution de deux suites causale est causale
\item La composition de deux SLI RIF est RIF 
\end{itemize}
\end{pop}

\begin{defn}[Transformée en Z]
\textit{Si h est un signal défini sur Z et est sommable. On apelle transformée en Z de h, la fonction H défini sur le cercle unité de C} $$H(z) = \sum_{n\in Z} h_{n}z^{-n}$$
\end{defn}

\begin{pop}
\textit{Si h est une suite sommable et que H est sa transformée en Z , alors on a :}
$$h_{n}= \int_{-1/2}^{1/2}H(e^{2i\pi \nu})e^{2i\pi \nu n}\mathrm{d}\nu$$
\end{pop}

\begin{pop}
\textit{Si ($x_{n}$) et ($y_{n}$) sont deux signaux sommables, on note X et Y leurs transformée en Z, on note u la convolution de x et y, on a :}$$U(z)=X(z)Y(z)$$
\end{pop}

\begin{pop}
\textit{Si T est un SLI récursif stable et h sa réponse impulsionnelle, H admet une transformée en Z notée H :}
$$H(z) = \dfrac{P(z^{-1})}{Q(z^{-1})} $$
\end{pop}

\begin{defn} On appelle \textbf{zeros} du filtre les zéros de la fonction $P(z^{-1})$, c'est à dire les inverse du polynôme P. On appelle \textbf{pôles}, les zéros de $Q(z^{-1})$.
\end{defn}

