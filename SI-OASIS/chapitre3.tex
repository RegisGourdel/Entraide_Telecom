\begin{voc}
	\textbf{Causalité}
	\begin{itemize}
	\item $h$ est causale si $\forall n<0, h_{n}=0$.
	\item Un SLI est causal si sa réponse impulsionnelle est causale.
	\item Une suite $h$ est anti-causale si $\forall h \geq 0, h_n = 0$ et un SLI est anti-causal si sa RI l'est.
	\item Suite bilatère : qui n'est ni causale, ni anti-causale.
	\item RIF : un SLI à réponse impulsionnelle finie.
	\item RII : un SLI à réponse impulsionnelle infinie.
	\end{itemize}
\end{voc}

\begin{pop}
	\begin{itemize}
	\item La convolution de deux suites causale est causale.
	\item La composition de deux suites à support fini est une suite à support fini.
	\item La composition de deux SLI causaux est causale.
	\item La composition de deux SLI RIF est RIF.
	\end{itemize}
\end{pop}

\begin{defn}
	Si $h$ est un signal défini sur $\Z$ et est sommable. On appelle \textbf{transformée en $Z$} de $h$, la fonction $H$ défini sur le cercle unité de $\C$ par
	\vspace{-0.15cm}$$
	H(z) = \sum_{n\in Z} h_{n}z^{-n}\ .
	\vspace{-0.15cm}$$
\end{defn}

\begin{pop}
	Si $h$ est une suite sommable et que $H$ est sa transformée en $Z$ , alors on a :
	\vspace{-0.15cm}$$
	\forall n \in \Z, h_{n}= \int_{-1/2}^{1/2} H \left( e^{2i\pi \nu} \right) e^{2i\pi \nu n} \diff \nu\ .
	\vspace{-0.15cm}$$
\end{pop}

\begin{pop}
	Si ($x_{n}$) et ($y_{n}$) sont deux signaux sommables, on note $X$ et $Y$ leurs transformée en $Z$, on note $u$ la convolution de $x$ et $y$, on a :
	\vspace{-0.15cm}$$
	U(z)=X(z)Y(z)\ .
	\vspace{-0.15cm}$$
\end{pop}

\begin{defn}[\textbf{Filtres récursifs stables}]
	Un SLI sur $\Z$ est dit récursif stable s'il vérifie les conditions suivantes :
	\begin{enumerate}
	\item
		Sa réponse impulsionnelle est sommable ($\sum_n \abs{h_n} < + \infty$).
	\item
		Il existe des coefficients $a_0,\ldots,a_p$ et $b_0,\ldots,b_q$ tels que, si $(x_n)_n$ est une entrée et $(y_n)_n$ la sortie qui lui correspond par le SLI, alors
		\vspace{-0.15cm}$$
		\forall n \in \Z, b_0 y_n + b_1 y_{n - 1} + \ldots + b_q y_{n - q} = a_0 x_n + a_1 x_{n - 1} + \ldots + a_p x_{n - p}
		\vspace{-0.15cm}$$
		Les $a_i$ et $b_j$ sont appelés coefficients du SLI.
	\item
		Les polynômes $\sum_i a_i z^i$ et $\sum_i b_i z^i$ sont premiers entre eux.
	\end{enumerate}
\end{defn}

\begin{pop}
	Si $T$ est un SLI récursif stable et $h$ sa réponse impulsionnelle, $H$ admet une transformée en $Z$ notée $H$ :
	\vspace{-0.15cm}$$
	H(z) = \frac{P(z^{-1})}{Q(z^{-1})}\ .
	\vspace{-0.15cm}$$
\end{pop}

\begin{defn}
	On appelle \textbf{zéros} du filtre les zéros de la fonction $P(z^{-1})$, c'est à dire les inverses du polynôme $P$.
	On appelle \textbf{pôles}, les zéros de $Q(z^{-1})$.
\end{defn}

\begin{pop}
	Un SLI récursif stable dont l'équation de récurrence est
	\vspace{-0.15cm}$$
	\forall n \in \Z, y_n + b_1 y_{n - 1} + \ldots + b_q y_{n - q} = a_0 x_n + a_1 x_{n - 1} + \ldots + a_p x_{n - p}
	\vspace{-0.15cm}$$
	est causal si et seulement si tous ses pôles sont dans l'intérieur du disque unité (i.e. de module strictement plus petit que $1$).
\end{pop}

\begin{defn}
	\textbf{Filtre à minimum de phase} : filtre récursif stable causal dont l'inverse est aussi stable et causal.
	Cela est équivalent à dire que ses pôles et ses zéros sont dans l'intérieur du disque unité.
\end{defn}
